\newpage % Rozdziały zaczynamy od nowej strony.
\section{Wprowadzenie}

\subsection{Cel pracy}

Uzyskanie informacji o środowisku wewnątrz budynków poprzez:
\begin{itemize}
    \item klasyfikację pomieszczenia
    \item segmentację semantyczną
\end{itemize} 
% Celem pracy inżynierskiej są dwa zadania:
% \begin{itemize}
%     \item segmentacja środowiska wewnątrz budynku
%     \item klasyfikacja pomieszczeń
% \end{itemize}
% \subsection{Założenia}
% Praca zakłada wykonanie celów pracy w środowisku wewnątrz budynków, co więcej będzie to środowisko domowe. Ponadto inferencja zostanie przeprowadzona na robocie Tiago, który jest wyposażony w kamerę Kinect.
\subsection{Motywacje}
Istnieje wiele powodów, dla których temat pracy jest wart uwagi.

Po pierwsze rozwiązanie może być wykorzystane w nawigacji robota. Wykrywanie przeszkód jest kluczowym aspektem możliwości poruszania się robota. Zostanie ono podjęte przez zadanie segmentacji. Należy zwrócić uwagę, że robot powinien zachowywać się ostrożniej w kuchni oraz w łazience. Ta informacja zostanie uzyskana poprzez klasyfikację sceny.

Innym zastosowanie rozważanego rozwiązania jest pomoc dla osób niewidomych. Osoba niepełnosprawna mogłaby wówczas poruszać się po środowisku domowym z większą łatwością, mając na sobie kamerę oraz informację o otaczającej przestrzeni.
