\newpage % Rozdziały zaczynamy od nowej strony.
\section{Wprowadzenie}

\subsection{Cel pracy}
Celem pracy jest zbadanie problemu wspólnej segmentacji semantycznej i klasyfikacji sceny we wnętrzach. Segmentacja semantyczna polega na przypisaniu etykiety do każdego piksela obrazu, natomiast klasyfikacja sceny polega na rozpoznaniu typu sceny przedstawionej na obrazie. Oba zadania mają szerokie spektrum zastosowań, takich jak autonomiczna nawigacja czy robotyka manipulacyjna.

Środowiska wewnętrzne, takie jak domy i biura, stanowią unikalny zestaw wyzwań dla segmentacji semantycznej i klasyfikacji scen. Środowiska te są często nieuporządkowane i~zawierają wiele różnych obiektów, co utrudnia dokładną segmentację i klasyfikację. Dodatkowo wnętrza mogą się znacznie różnić pod względem układu i wyglądu, co czyni trudnym opracowanie modelu, który może być uogólniony na różne typy scen wewnętrznych.

Głównym celem tej pracy jest opracowanie modelu opartego na głębokim uczeniu przy jednoczesnej semantycznej segmentacji i klasyfikacji sceny w różnych rodzajach pomieszczeń. Proponowany model zostanie wytrenowany i oceniony na dużym zbiorze danych scen wewnętrznych i zostanie porównany z aktualnymi metodami segmentacji semantycznej i klasyfikacji scen.

Aby osiągnąć ten cel, zostaną podjęte następujące pytania badawcze
\begin{itemize}
    \item Jak można zaprojektować model oparty na głębokim uczeniu do wspólnej segmentacji semantycznej i klasyfikacji scen w środowiskach wewnętrznych?
    \item Czy przestrzeń reprezentacji po wytrenowaniu na zadaniu segmentacji semantycznej może być użyta do zadania klasyfikacji sceny?
    \item Jak dobrze proponowany model radzi sobie na dużym zbiorze danych scen wewnętrznych i jak wypada w porównaniu z aktualnymi metodami segmentacji semantycznej i klasyfikacji scen osobno?
    \item Jak proponowany model może być wykorzystany do poprawy wydajności w robotyce mobilnej?
\end{itemize}

Podsumowując, celem tej pracy jest opracowanie i ocena modelu opartego o głębokie uczenie dla wspólnej segmentacji semantycznej i klasyfikacji scen w środowiskach wewnętrznych oraz zbadanie potencjału modelu do poprawy jakości i wydajności na tychże zadaniach.

\subsection{Motywacje}
Wspólna segmentacja oraz klasyfikacja polega na oznaczaniu i kategoryzowaniu różnych regionów w obrębie wnętrz, oraz ich charakteryzowanie. Techniki te mogą być stosowane w różnych dziedzinach, w tym w robotyce, zarządzaniu budynkami i rozszerzonej rzeczywistości.

W robotyce wspólna segmentacja semantyczna i klasyfikacja scen może być wykorzystana do umożliwienia robotom zrozumienia i nawigacji w środowiskach wewnętrznych. Może to obejmować identyfikację różnych obiektów i regionów w scenie, takich jak ściany, meble i ludzie, a także określenie ogólnego układu i funkcjonalności przestrzeni, np. czy jest to kuchnia, czy salon. Dzięki zrozumieniu środowiska w ten sposób roboty mogą poprawić swoją zdolność do wykonywania zadań, takich jak manipulacja obiektami, nawigacja i interakcja człowiek-robot.

W zarządzaniu budynkiem wspólna segmentacja semantyczna i klasyfikacja sceny może być wykorzystana do poprawy funkcjonalności i wydajności budynków poprzez automatyczną identyfikację i etykietowanie różnych obiektów i regionów w budynku. Może to obejmować identyfikację różnych pomieszczeń, klatek schodowych i wind, jak również określenie ogólnego układu i funkcjonalności przestrzeni, np. czy jest to biuro, czy fabryka. Dzięki zrozumieniu środowiska w ten sposób, systemy zarządzania budynkiem mogą poprawić swoją zdolność do wykonywania zadań, takich jak zarządzanie energią, bezpieczeństwo i wykrywanie zajętości.

W dziedzinie rozszerzonej rzeczywistości wspólna segmentacja semantyczna i klasyfikacja sceny mogą być wykorzystane do poprawy realizmu doświadczeń AR poprzez zrozumienie środowiska rzeczywistego i rozszerzenie go o dodatkowe informacje lub obiekty wirtualne. Dzięki zrozumieniu środowiska w ten sposób, doświadczenia AR mogą być bardziej świadome kontekstowo, zapewniając w ten sposób bardziej realistyczne i~angażujące doświadczenia.

Wspólna segmentacja semantyczna i klasyfikacja sceny w środowiskach wewnętrznych jest wymagającym, ale ważnym obszarem badawczym o wielu potencjalnych zastosowaniach. Wiąże się to z wykorzystaniem zaawansowanych technik widzenia komputerowego, solidnych i wydajnych algorytmów oraz starannej oceny w rzeczywistych środowiskach wewnętrznych. W miarę rozwoju technologii prawdopodobnie zostaną zidentyfikowane nowe przypadki użycia i zastosowania, i nadal będzie to aktywny obszar badań.
