\clearpage
\newpage % Rozdziały zaczynamy od nowej strony.
\section{Podsumowanie}

W tym rozdziale analizie poddany będzie całokształt pracy. Przedstawione zostaną wnioski z wynikające z poprzedniego rozdziału. Na końcu praca zostanie podsumowana, przedstawiając ostateczne konkluzje.

\subsection{Wnioski}

Środowiska wewnętrzne to unikalny zestaw wyzwań dla zadania łącznej klasyfikacji oraz segmentacji semantycznej. Wiele różnych elementów w wielu skalach stanowi wyzwanie nawet dla najnowocześniejszych algorytmów sztucznej inteligencji. Nie można ukryć też faktu, że różnice w wyglądzie pomieszczeń nie sprzyjają osiąganiu wysokich wyników. Szczególnie utrudniające okazało się samo oznaczenie danych, które czasem było sprzeczne lub nieodpowiednie. Występował tutaj szereg problemów. Po pierwsze dla zadania segmentacji etykiety takie jak obiekt były bardzo trudne do rozwiązania. Za tą nazwą kryło się wszystko, co nie mieściło się w ramach innych etykiet. Nie jest to pożądane, ponieważ trudno znaleźć wspólną reprezentację dla tak szeroko pojętej etykiety. Innym problem jest występowanie klasy stół oraz mebel. Zachodzi pytanie, czy stół nie jest meblem? Z analizy przykładów wynikało, że klasy te często były mylone. Inny problem okazały się pojedyncze przykłady ze zbioru danych, takie jak regał z książkami. Model zaznacza regał jako książki, co jest rzeczywiście prawdą, gdyż tam znajdowały się książki. Do klasy mebel powinno zaliczyć się tylko drewniane części regału, a nie koniecznie jego zawartość. Innym przykładem jest zaznaczenie plakatu mapy, który został zaklasyfikowany przez autorów jako obiekt, a nie konkretnie obraz.

Pomimo tych wyzwań, przedstawiona w niniejszej pracy architektura uczenia wielozadaniowego wykazała zdolność do skutecznego radzenia sobie z tymi trudnościami. Nie można stwierdzić, że uczenie wielozadaniowe pod każdym aspektem osiąga najwyższe rezultaty, jednak uważam, że w wielu zastosowaniach będzie to najbardziej optymalne rozwiązanie. Uczenie wyłącznie klasyfikacji czy segmentacji pozwoliło porównać rezultaty uzyskane na pojedynczych sieciach z architekturą wielozadaniową.

Makrośrednie miary jakości pokazały, że w przypadku segmentacji semantycznej najlepsze jest podejście uczenia samej segmentacji. Uczenie wielozadaniowe jest nieznacznie gorsze. Jednakże uczenie wielozadaniowe osiągnęło lepszy rezultat w zadaniu klasyfikacji sceny niż inne metody uczenia w tym uczenie wyłącznie klasyfikacji. W przytaczanych artykułach (\cite{mehta2018net}, \cite{9892852}) autorzy osiągają tę samą dokładność na segmentacji i znaczne lepsze rezultaty dla zadania klasyfikacji ucząc te zadania łącznie. Uważam, że na tej podstawie można twierdzić, że uczenie wielozadaniowe konkretnie dla zadania segmentacji semantycznej i klasyfikacji sceny dla domeny scen wnętrz wpływa dodatnio na zadanie klasyfikacji. Zadanie segmentacji semantycznej dostarcza dużo więcej informacji dla funkcji straty niż klasyfikacja sceny, gdyż w pierwszym przypadku klasyfikujemy każdy piksel na obrazie, a w drugim cały obraz. W mojej opinii zachodzi tutaj rozszerzenie zbioru danych oraz tak zwane podsłuchiwanie, o których pisał Ruder \cite{ruder2017overview}. Segmentacja semantyczna dostarcza zadaniu klasyfikacji dodatkowe informacje, które ostatecznie polepszają jakość modelu.

Uczenie wielozadaniowe jest bardzo korzystne z punktu widzenia wydajności czasowej. Model korzysta z 2 razy mniej parametrów. Wpłynęło to bezpośrednio na czas uczenia oraz co najważniejsze wnioskowania. Oczywiste jest, że urządzenia IoT czy robotyka mobilna ograniczają zasoby sprzętowe. Jednocześnie kluczowa czy jest tam czas reakcji. Dwukrotne przyśpieszenie jest zatem tym cenniejsze.

Oprócz wniosków bezpośrednio wynikających z uczenia wielozadaniowego, podczas eksperymentów można było zaobserwować różne ciekawe zdarzenia. Ciekawym okazał się transfer wiedzy, który wypadał najsłabiej. Nie jest to dziwne. Baza ImageNet znacząco odbiega od prezentowanych obrazów pomieszczeń. Przede wszystkim obrazy z tej bazy są skoncentrowane na pojedynczym obiekcie. Zdjęcia pomieszczeń to zdjęcie scen, a~więc wielu obiektów rozdystrybuowanych na całym obrazie. Poza tym rozkład związany z~pomieszczeniami jest długoogonowy. Większość obrazu jest często zdominowana przez podłogę czy ściany.

Innym eksperymentem było uczenie pośrednio oraz bezpośrednio z segmentacji. W~pierwszym przypadku skorzystano z wyjścia enkodera jako przestrzeni reprezentacji. Wyniki były wprawdzie gorsze niż uczenia wielozadaniowego, ale prawie tak samo dobre, jak w przypadku uczenia wyłącznie klasyfikacji. Oznacza to, że przestrzeń reprezentacji wygenerowana przez segmentację semantyczną nadaje się i może być stosowana do wyuczenia klasyfikacji. Przypadek bezpośredniej klasyfikacji z segmentacji jest najgorszym. Przestrzeń reprezentacji składająca się wyłącznie z 13 kanałów to zdecydowanie za mało, by reprezentować sceny.

\subsection{Podsumowanie}

W pracy opracowano model oparty o głębokie uczenie, który jednocześnie segmentował semantycznie pomieszczenia oraz przyporządkowywał im rodzaj sceny. Udało się zrealizować łączne uczenie obu zadań. Co więcej, uczenie wielozadaniowe okazało się często bardzo korzystnym rozwiązaniem w porównaniu z metodami klasycznymi. Szczególnie cenne we wspólnej architekturze jest wydajność czasowa, która wpływa na czas uczenia oraz wnioskowanie modelu. Przestrzeń reprezentacji enkodera po wytrenowaniu segmentacji semantycznej może z dużą skutecznością być użyta do zadania klasyfikacji sceny. Korzystanie z metod finetungu i bezpośredniej klasyfikacji nie przynosi oczekiwanych rezultatów.

Cel pracy został zrealizowany z dużym powodzeniem. Co więcej, uzyskana została odpowiedź na pytania badawcze postawione w pierwszej jej części. Uważam, że wiele scenariuszy testowych pozwoliło rzetelnie odpowiedzieć na wątpliwości związane z jakością wspólnego uczenia, pokazując jej zalety oraz ograniczenia.
% \subsection{Dyskusja}

% -mozna bylo lepiej to zrobic:
%     - augmentacja
%     - więcej epok
%     - regularyzacja smoothingiem
%     - regularayzacja weight decay
% -zeby w pelni moc ocenic potencjal uczenia wielozadaniwoe nalezaloby spawdzic modele na wielu zbiorach, w razie potrzeby zachowujac przestrzen reprezentacji a dosziflowujac ostatnie warstwy decyzyjbedroomeabedroom



% niepoprawen klasy szum podczas uczenia
% klasa obiekt generuja problemy klasy szerokie generuja problemy
% Analiza przewidywań dostarczyła wielu cennych szczegółów, które byłoby trundo zauważyć patrząc jedynie na liczby.
% drewno to meble
% obrazy zawsze były na powierzchniach takich jak sciany czy szafki
% REZULATAT
% słabe wyniki bo
% - sprzecznośc klas (regał mebel czy ksiązki, czy mebel i stół)
% - klasa objects


% odpowiedzenie na pytania badawcze



% 1. Spełniono założenia pracy
% 2. Odppowiedź na pytania badawcze 

% 1. Wnioski
% 2. Dyskusja nad rozwiązaniem
% Mimo osiągniecia oczekiwanych rezultatów, zawsze jest miejsce na poprawki. 
