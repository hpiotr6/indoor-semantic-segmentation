\newpage % Rozdziały zaczynamy od nowej strony.
\section{Rozwiązanie}
Zadania wizji komputerowej moga zostać podjęta na wiele sposobów. Szczególnie iteresujące są podjeścia do zadania łącznej segmentacji semantycznej i klasyfikacji sceny we wnętrzach. W tym rozdziale przedstawione zostaną wybrane metody, które zostały sprawdzone w ramach analizy problemu. W swoich rozważaniach będę bezpośrednio odnosił się do pytań badawczych postawionych w celu pracy, a więc: 

\begin{itemize}
    \item Jak można zaprojektować model oparty na głębokim uczeniu do wspólnej segmentacji semantycznej i klasyfikacji scen w środowiskach wewnętrznych?
    \item Czy przestrzeń reprezentacji po wytrenowaniu na zadaniu segmentacji semantycznej może być użyta do zadania klasyfikacji sceny?
    \item Jak dobrze proponowany model radzi sobie na dużym zbiorze danych scen wewnętrznych i jak wypada w porównaniu z aktualnymi metodami segmentacji semantycznej i klasyfikacji scen osobno?
    \item Jak proponowany model może być wykorzystany do poprawy wydajności w robotyce mobilnej?
\end{itemize}
Opis rozwiązań problemu zostanie poprzedzony przeglądem rozwiązań. Analiza dotychczasowych pozwoli lepiej ukierunkować badania. Korzystając z doświadczenia innych, będzie można wyrobić sobie intuicję, która pomoże podejmować konkretne decyzje.

\subsection{Przegląd rozwiązań}
Przegląd literatury jest kluczowym aspektem każdej pracy naukowej. W tym rozdziale zostaną przedstawione wyłącznie rozwiązania obejmujące łączną segmentację semantyczną oraz klasyfikację sceny. Szczególny nacisk położony zostanie na architektury głębokich sieci neuronowych z dogłębną analizą przepływu inferencji przez nie.
\vspace{0.5cm}
Niestety przyjetę założenia w pracy nie zostały opisane przez nikogo wczesniej, zgodnie z najlepszą wiedzą autora. Niektóre prace naukowe przedstawiają ten sam problem to jest klasyfikacji i segmentacji łącznie, ale obejmują go w innej domenie danych. Z drugiej artykuły obejmujące środwiska wnętrz są dobrze zdefiniowane, jednak często w swoich rozwiązaniach korzystają z obrazu głębki, który nie zawiera się w zakresie badań tej pracy. Nie mniej wszystkie poniższe artykuły stanowią cenne źródło informacji, które należy mniej lub bardziej dostoswać do rozważanego problemu.

Pierwszym z prezentowanych artykułów jest ,,Describing the Scene as a Whole: Joint Object Detection, Scene Classification and Semantic Segmentation'' autorstwa Yao j. et al (2012)\cite{yao2012describing}. Prezentuje on algorytm, który ówcześnie wyznaczył najlepsze podejście (ang. state-of-the-art (SOTA)). Autorzy wskajzują tutaj, że połączenie rozważanych zadań okazało się owocne nie tylko pod względem jakości, ale również wydajności w kontekście czasowym. Yao J. et al zwracają uwagę na połączenie szeregowe, które niestety propaguje błąd w kolejnych zadaniach, a było dotychczasowo szeroko stosowane. W swojej pracy wykorzysztują podejście równolegle zgodne z rysunkiem \ref{fig:scene-as-a-whole}. Podsumowujac, ,,Describing the Scene as a Whole: Joint Object Detection, Scene Classification and Semantic Segmentation'' nie jest propozycją architektury głębokiej sieci. Wskazuje on na problemy z łączenie zadań szeregowo, jednocześnie udowadniając, że taka praktyka był ówcześnie stosowana, więc nie można jej wykluczyć.

    \begin{figure}[ht!]
        \includegraphics[width=\textwidth]{img/joint-segmentation-and-classification.png}
        \caption{Describing the Scene as a Whole: Joint Object Detection, Scene Classification and Semantic Segmentation (2012) \cite{yao2012describing}.}
        \label{fig:scene-as-a-whole}
    \end{figure}

\begin{figure}[ht!]
    \includegraphics[width=\textwidth]{global-local-features.png}
    \caption{Let there be Color!: Joint End-to-end Learning of Global and Local Image Priors for Automatic Image Colorization with Simultaneous Classification 2016 \cite{iizuka2016let}.}
    \label{fig:parrarel-arch}
\end{figure}

Współcześnie do zadań wizji komputerowej używa się głębokich sieci neuronowych z uwagi na ich duże zdolności generalizacji skomplikowanych przestrzeni. Celem każdej architektury jest odpowiednia ekstrakcja cech w sposób łatwo ekstrahowalny. Architektury różnią się zatem sposobem generalizacji, a dokładniej ułożeniem warstw i ich parametrów. W ramach przeglądu literatury pochylono się nad różnymi metodami łączenia zadania segmentacji i klasyfikacji, ponieważ zadanie postawione w pracy, co do wiedzy autora, nie zostało wcześniej rozwiązane podobnymi metodami.

Pierwszy artykuł ,,Let there be Color!: Joint End-to-end Learning of Global and Local Image Priors for Automatic Image Colorization with Simultaneous Classification 2016 \cite{iizuka2016let}'' rozwiązuje problem kolorowania obrazków jednak, przekształcony może być użyty w pracy. Tego można dokonać odrzucając ostatnią warstwę konkatenacji w części segmentacji (rys. \ref{fig:parrarel-arch}). Przedstawiona architektura symultanicznie ekstrahuję cechy globalne oraz średniego poziomu, które odpowiednio służą klasyfikacji oraz segmentacji.

\begin{figure}[ht!]
    \includegraphics[width=\textwidth]{y-net.png}
    \caption{Y-Net: Joint Segmentation and Classification for Diagnosis of Breast Biopsy Images 2018 \cite{mehta2018net}.}
    \label{fig:y-net}
\end{figure}

Kolejnym artykułem jest ,,Y-Net: Joint Segmentation and Classification for Diagnosis of Breast Biopsy Images 2018 \cite{mehta2018net}''. Jest to standardowa architektura segmentacji \texttt{U-Net} rozszerzona o gałąź klasyfikacyjną (rys. \ref{fig:y-net}). Rozwiązanie to jest na pewno ciekawe z punku widzenia modularności rozwiązania.

\subsection{Zarys rozwiązania problemu}
% * multitask Learning
% * jednozadaniowe
% * wielozadaniowe
W celu realizacji zadania zdecydowano się na architekturę (najbliższą Y-Netu) o wspólnym enkoderze i o osobnych głowach, służących do egzekwowania konkretnych zadań (rys. \ref{fig:cep_arch}). Decyzja podyktowana była względnie prostą implementacją rozszerzenia wielu modeli segmentacji semantycznej o dodatkową głowę klasyfikacyjną. Co więcej stwierdzono, że ograniczenie się tylko do jednego backbone'u jest niesłychanie korzystne, gdyż znacząco ogranicza ilość parametrów sieci, co bezpośrednio przekłada się m.in. na czas inferencji. Należy zwrócić uwagę na fakt, iż właściwie zdecydowana większość parametrów znajduje się własnie w enkoderze.

Mając na uwadze, że symultaniczne uczenie może negatywnie wpływać na jakość uczenia obu zadań, eksperymenty przeprowadzono etapowo. Pierwszym etapem było uczenie jednozadaniowe. Eksperymenty polegały na sprawdzeniu jakości segmentacji oraz klasyfikacji osobno. Wykorzystano do tego tę samą archtekturę, która używana była poźniej w drugim etapie. Mianowicie, mając dwie głowy każdorazowo zamrażano głowę nie biorącą udziału w uczeniu (rys. \ref{fig:arch-scene-seg}). Zapewnia to pewność posiadania tej samej architektury, a w szczególności rzetelne porównanie z etapem uczenia wielozadaniowego.

Drugim etapem było przeprowadzenie eksperymentów w uczeniu wielozadaniowym (rys. \ref{fig:arch-full}). Funkcja celu zdefiniowana była jako suma wartości funkcji celów dla obu zadań. W wyniku progpagacji wstecznej wagi aktualizowane były zgodnie z zagregowaną stratą.

Ostatecznie porównano jakość na przesztreni obu etapów.

\begin{figure}[ht!]
    \centering
    \includegraphics[width=0.75\textwidth]{cep_arch.png}
    \caption{Architektura sieci zastosowana w pracy inżynierskiej.}
    \label{fig:cep_arch}
\end{figure}

\begin{figure}[ht!]
    \centering
    \begin{subfigure}[b]{0.49\textwidth}
        \centering
        \includegraphics[width=\textwidth]{arch:scene.png}
        \caption{Architektura sieci wyłącznie w zadaniu klasyfikacji.}
    \end{subfigure}
    \hfill
    \begin{subfigure}[b]{0.49\textwidth}
        \centering
        \includegraphics[width=\textwidth]{arch:seg.png}
        \caption{Architektura sieci wyłącznie w zadaniu segmentacji semantycznej.}
    \end{subfigure}
    \caption[]{Podejście jednozadaniowe.}
    \label{fig:arch-scene-seg}
\end{figure}

\begin{figure}[ht!]
    \centering
    \includegraphics[width=0.75\textwidth]{arch:full.png}
    \caption{Architektura sieci jako uczenie wielozadaniowego.}
    \label{fig:arch-full}
\end{figure}