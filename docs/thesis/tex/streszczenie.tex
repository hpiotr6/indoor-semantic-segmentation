% Przetwarzanie obrazu jest zagadnieniem skomplikowanym. Cały czas prowadzone są badania mające na celu osiąganie coraz lepszych rezultatów w tej dziedzinie. Ma ona zresztą szereg zastowań, takich jak percepcja A
Przetwarzanie obrazu ma wiele zastosowań. Jednym z nich jest semantyczna analiza środowiska. W tej pracy inżynierskiej zostanie przeprowadzona semantyczna analiza środowiska we wnętrzach. Do jej wykonania pozyskane zostaną informacje o przynależności pikseli do znaczeniowych grup (segmentacja semantyczna) oraz każdemu obrazowi zostanie przyporządkowana nazwa miejsca, które przedstawia (klasyfikacja sceny). Środowiska wewnętrzne stanowią unikalny zestaw wyzwań. Jednocześnie rozwiązywanie ich oddzielnie wydaje się mało efektywne. Trudno wyobrazić sobie szeregowe odtwarzanie algorytmów w dziedzinie robotyki czy urządzeń Internetu Rzeczy, gdzie zasoby są bardzo ograniczone, a wymagania na szybkość odpowiedzi wysoko postawione. Implikuje to wybór zaawansowanych algorytmów, które najlepiej działają równolegle. Aktualnym trendem rozwiązań przetwarzania obrazu są głębokie sieci neuronowe, dzięki swojej wysokiej skuteczności. Co więcej, te sieci pozwalają się łączyć w architektury, umożliwiając jednoczesne wykonywanie obu wspomnianych wcześniej zadań. Takie podejście w literaturze jest określane jako uczenie wielozadaniowe. Ma ono szereg zalet. W tej pracy pokażę, że wspólna segmentacja semantyczna oraz klasyfikacja scen w środowiskach wewnętrznych jest korzystna. Dodatkowo zostaną przeprowadzone badania nad aktualnymi technikami głębokiego uczenia w kontekście uczenia wielu zadań. Ostatecznie wyniki wszystkich metod zostań porównane i ocenione na dużym zbiorze danych scen wnętrz.



% Często jest związana w jednoczesnym dostarczaniem wielu informacji. W praktyce oznacza to wykorzystanie wielu algorytmów, by sprostać powyższym wymaganiom. Nie mniej jedank korzytsanie z wielu alogrytmów jest często nieefektywne. Trudno wyobrazić sobie szeregowe odtwarzanie algorytmów w dziedzinie robotyki czy urządzeń Internetu Rzeczy, gdzie zasoby są bardzo ograniczone. Jednocześnie czas reakcji ma kluczowe znaczenie.