Image processing has many applications. One of them is a semantic analysis of the environment. In this thesis, a semantic analysis of the environment in the interior will be carried out. In order to perform it, information about the belonging of pixels to meaningful groups will be extracted (semantic segmentation), and each image will be assigned the name of the place it represents (scene classification). Indoor environments present a unique set of challenges. At the same time, solving them separately seems inefficient. It is difficult to imagine serial algorithmic reproduction in robotics or Internet of Things devices, where resources are minimal, and demands on response speed are high. This implies the selection of advanced algorithms that work best in parallel. The current trends in image processing solutions are deep neural networks, thanks to their high efficiency. Moreover, these networks can be combined into architectures, enabling the previously mentioned tasks to be performed simultaneously. This approach is referred to in the literature as multi-task learning. It has several advantages. In this thesis, joint semantic segmentation and scene classification in indoor environments will be shown as beneficial. In addition, research will be conducted on current deep learning techniques in the context of multi-task learning. Finally, the results of all methods will be compared and evaluated on a large dataset of indoor scenes.