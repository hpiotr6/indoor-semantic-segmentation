%%%%%%%%%%%%%%%%%%%%%%%%%%%%%%%%%%%%%%%%%%%%%%%%%%%%%%%
%% Bachelor's & Master's Thesis Template             %%
%% Copyleft by Artur M. Brodzki & Piotr Woźniak      %%
%% Faculty of Electronics and Information Technology %%
%% Warsaw University of Technology, 2019-2020        %%
%%%%%%%%%%%%%%%%%%%%%%%%%%%%%%%%%%%%%%%%%%%%%%%%%%%%%%%

\documentclass[
    left=2.5cm,         % Sadly, generic margin parameter
    right=2.5cm,        % doesnt't work, as it is
    top=2.5cm,          % superseded by more specific
    bottom=3cm,         % left...bottom parameters.
    bindingoffset=6mm,  % Optional binding offset.
    nohyphenation=false % You may turn off hyphenation, if don't like.
]{eiti/eiti-thesis}

\langpol % Dla języka angielskiego mamy \langeng
\graphicspath{{img/}}             % Katalog z obrazkami.
\addbibresource{bibliografia.bib} % Plik .bib z bibliografią

\usepackage{caption}
\usepackage{subcaption}
% \usepackage{csquotes}

\begin{document}

%--------------------------------------
% Strona tytułowa
%--------------------------------------
% \MasterThesis % Dla pracy inżynierskiej mamy \EngineerThesis
\EngineerThesis
\instytut{Automatyki i Informatyki Stosowanej}
\kierunek{Automatyka i Robotyka}
% \specjalnosc{XXXXXX}
\title{
   Semantyczna analiza środowiska przez robota usługowego 
}
\engtitle{ % Tytuł po angielsku do angielskiego streszczenia
    Semantic analysis of the environment by a mobile service robot
}
\author{Piotr Hondra}
\album{303752}
\promotor{mgr inż. Maciej Stefańczyk}
\date{2023}
\maketitle

%--------------------------------------
% Streszczenie po polsku
%--------------------------------------
\cleardoublepage % Zaczynamy od nieparzystej strony
\streszczenie
% Przetwarzanie obrazu jest zagadnieniem skomplikowanym. Cały czas prowadzone są badania mające na celu osiąganie coraz lepszych rezultatów w tej dziedzinie. Ma ona zresztą szereg zastowań, takich jak percepcja A
Przetwarzanie obrazu ma wiele zastosowań. Jednym z nich jest semantyczna analiza środowiska. W tej pracy inżynierskiej zostanie przeprowadzona semantyczna analiza środowiska we wnętrzach. Do jej wykonania pozyskane zostaną informacje o~przynależności pikseli do znaczeniowych grup (segmentacja semantyczna) oraz każdemu obrazowi zostanie przyporządkowana nazwa miejsca, które przedstawia (klasyfikacja sceny). Środowiska wewnętrzne stanowią unikalny zestaw wyzwań. Jednocześnie rozwiązywanie ich oddzielnie wydaje się mało efektywne. Trudno wyobrazić sobie szeregowe odtwarzanie algorytmów w dziedzinie robotyki czy urządzeń Internetu Rzeczy, gdzie zasoby są bardzo ograniczone, a wymagania na szybkość odpowiedzi wysoko postawione. Implikuje to wybór zaawansowanych algorytmów, które najlepiej działają równolegle. Aktualnym trendem rozwiązań przetwarzania obrazu są głębokie sieci neuronowe, dzięki swojej wysokiej skuteczności. Co więcej, te sieci pozwalają się łączyć w architektury, umożliwiając jednoczesne wykonywanie obu wspomnianych wcześniej zadań. Takie podejście w literaturze jest określane jako uczenie wielozadaniowe. Ma ono szereg zalet. W tej pracy pokażę, że wspólna segmentacja semantyczna oraz klasyfikacja scen w środowiskach wewnętrznych jest korzystna. Dodatkowo zostaną przeprowadzone badania nad aktualnymi technikami głębokiego uczenia w kontekście uczenia wielu zadań. Ostatecznie wyniki wszystkich metod zostaną porównane i ocenione na dużym zbiorze danych scen wnętrz.



% Często jest związana w jednoczesnym dostarczaniem wielu informacji. W praktyce oznacza to wykorzystanie wielu algorytmów, by sprostać powyższym wymaganiom. Nie mniej jedank korzytsanie z wielu alogrytmów jest często nieefektywne. Trudno wyobrazić sobie szeregowe odtwarzanie algorytmów w dziedzinie robotyki czy urządzeń Internetu Rzeczy, gdzie zasoby są bardzo ograniczone. Jednocześnie czas reakcji ma kluczowe znaczenie.
\slowakluczowe głębokie sieci neuronowe, uczenie wielozadaniowe, segmentacja semantyczna, klasyfikacja sceny

%--------------------------------------
% Streszczenie po angielsku
%--------------------------------------
\newpage
\abstract
Image processing has many applications. One of them is a semantic analysis of the environment. In this thesis, a semantic analysis of the environment in the interior will be carried out. In order to perform it, information about the belonging of pixels to meaningful groups will be extracted (semantic segmentation), and each image will be assigned the name of the place it represents (scene classification). Indoor environments present a unique set of challenges. At the same time, solving them separately seems inefficient. It is difficult to imagine serial algorithmic reproduction in robotics or Internet of Things devices, where resources are minimal, and demands on response speed are high. This implies the selection of advanced algorithms that work best in parallel. The current trends in image processing solutions are deep neural networks, thanks to their high efficiency. Moreover, these networks can be combined into architectures, enabling the previously mentioned tasks to be performed simultaneously. This approach is referred to in the literature as multi-task learning. It has several advantages. In this thesis, joint semantic segmentation and scene classification in indoor environments will be shown as beneficial. In addition, research will be conducted on current deep learning techniques in the context of multi-task learning. Finally, the results of all methods will be compared and evaluated on a large dataset of indoor scenes.
\keywords deep neural networks, multi-task learning, semantic segmentation, scene classification

%--------------------------------------
% Oświadczenie o autorstwie
%--------------------------------------
% \cleardoublepage  % Zaczynamy od nieparzystej strony
% \pagestyle{plain}
% \makeauthorship

%--------------------------------------
% Spis treści
%--------------------------------------
\cleardoublepage % Zaczynamy od nieparzystej strony
\tableofcontents

%--------------------------------------
% Rozdziały
%--------------------------------------
\cleardoublepage % Zaczynamy od nieparzystej strony
\pagestyle{headings}

\newpage % Rozdziały zaczynamy od nowej strony.
\section{Wprowadzenie}

\subsection{Cel pracy}
Celem pracy jest zbadanie problemu wspólnej segmentacji semantycznej i klasyfikacji sceny w we wnętrzach. Segmentacja semantyczna polega na przypisaniu etykiety do każdego piksela obrazu, natomiast klasyfikacja sceny polega na rozpoznaniu typu sceny przedstawionej na obrazie. Oba zadania mają szerokie spektrum zastosowań, takich jak autonomiczna nawigacja czy robotyka manipulacyjna.

Środowiska wewnętrzne, takie jak domy i biura, stanowią unikalny zestaw wyzwań dla segmentacji semantycznej i klasyfikacji scen. Środowiska te są często nieuporządkowane i zawierają wiele różnych obiektów, co utrudnia dokładną segmentację i klasyfikację. Dodatkowo wnętrza mogą się znacznie różnić pod względem układu i wyglądu, co czyni trudnym opracowanie modelu, który może być uogólniony na różne typy scen wewnętrznych.

Głównym celem tej pracy jest opracowanie modelu opartego na głębokim uczeniu przy jednoczesnej semantycznej segmentacji i klasyfikacji sceny w różnych rodzajach pomieszczeń. Proponowany model zostanie wytrenowany i oceniony na dużym zbiorze danych scen wewnętrznych i zostanie porównany z aktualnymi metodami segmentacji semantycznej i klasyfikacji scen.

Aby osiągnąć ten cel, zostaną podjęte następujące pytania badawcze
\begin{itemize}
    \item Jak można zaprojektować model oparty na głębokim uczeniu do wspólnej segmentacji semantycznej i klasyfikacji scen w środowiskach wewnętrznych?
    \item Czy przestrzeń reprezentacji po wytrenowaniu na zadaniu segmentacji semantycznej może być użyta do zadania klasyfikacji sceny?
    \item Jak dobrze proponowany model radzi sobie na dużym zbiorze danych scen wewnętrznych i jak wypada w porównaniu z aktualnymi metodami segmentacji semantycznej i klasyfikacji scen osobno?
    \item Jak proponowany model może być wykorzystany do poprawy wydajności w robotyce mobilnej?
\end{itemize}

Podsumowując, celem tej pracy jest opracowanie i ocena modelu opartego o głębokim uczeniu dla wspólnej segmentacji semantycznej i klasyfikacji scen w środowiskach wewnętrznych oraz dalsze badanie potencjału modelu do poprawy innych zadań rozumienia scen wewnętrznych.

\subsection{Motywacje}
% Istnieje wiele powodów, dla których temat pracy jest wart uwagi.

% Po pierwsze rozwiązanie może być wykorzystane w nawigacji robota. Wykrywanie przeszkód jest kluczowym aspektem możliwości poruszania się robota. Zostanie ono podjęte przez zadanie segmentacji. Należy zwrócić uwagę, że robot powinien zachowywać się ostrożniej w kuchni oraz w łazience. Ta informacja zostanie uzyskana poprzez klasyfikację sceny.

% Innym zastosowanie rozważanego rozwiązania jest pomoc dla osób niewidomych. Osoba niepełnosprawna mogłaby wówczas poruszać się po środowisku domowym z większą łatwością, mając na sobie kamerę oraz informację o otaczającej przestrzeni.
Wspólna segmentacja oraz klasyfikacja polega na oznaczaniu i kategoryzowaniu różnych regionów w obrębie wnętrz, natomiast klasyfikacja sceny polega na określeniu ogólnego układu i funkcjonalności przestrzeni. Techniki te mogą być stosowane w różnych dziedzinach, w tym w robotyce, inteligentnych domach, zarządzaniu budynkami i rozszerzonej rzeczywistości.

Robotyka:
W robotyce, wspólna segmentacja semantyczna i klasyfikacja scen może być wykorzystana do umożliwienia robotom zrozumienia i nawigacji w środowiskach wewnętrznych. Może to obejmować identyfikację różnych obiektów i regionów w scenie, takich jak ściany, meble i ludzie, a także określenie ogólnego układu i funkcjonalności przestrzeni, np. czy jest to kuchnia czy salon. Dzięki zrozumieniu środowiska w ten sposób, roboty mogą poprawić swoją zdolność do wykonywania zadań, takich jak manipulacja obiektami, nawigacja i interakcja człowiek-robot.

Inteligentne domy:
Wspólna segmentacja semantyczna i klasyfikacja sceny mogą być również wykorzystane do poprawy funkcjonalności inteligentnych domów. Na przykład, techniki te mogą być wykorzystywane do automatycznej identyfikacji i etykietowania różnych obiektów i regionów w domu, takich jak meble, urządzenia i inne obiekty. Dodatkowo techniki te mogą być wykorzystane do określenia ogólnego układu i funkcjonalności przestrzeni, np. czy jest to sypialnia czy jadalnia. Dzięki zrozumieniu środowiska w ten sposób, inteligentne domy mogą poprawić swoją zdolność do wykonywania zadań, takich jak kontrola oświetlenia, zarządzanie energią i automatyka domowa.

Zarządzanie budynkiem:
W zarządzaniu budynkiem, wspólna segmentacja semantyczna i klasyfikacja sceny może być wykorzystana do poprawy funkcjonalności i wydajności budynków poprzez automatyczną identyfikację i etykietowanie różnych obiektów i regionów w budynku. Może to obejmować identyfikację różnych pomieszczeń, klatek schodowych i wind, jak również określenie ogólnego układu i funkcjonalności przestrzeni, np. czy jest to biuro czy fabryka. Dzięki zrozumieniu środowiska w ten sposób, systemy zarządzania budynkiem mogą poprawić swoją zdolność do wykonywania zadań, takich jak zarządzanie energią, bezpieczeństwo i wykrywanie zajętości.

Augmented Reality (rozszerzona rzeczywistość):
W dziedzinie rozszerzonej rzeczywistości, wspólna segmentacja semantyczna i klasyfikacja sceny mogą być wykorzystane do poprawy realizmu doświadczeń AR poprzez zrozumienie środowiska rzeczywistego i rozszerzenie go o dodatkowe informacje lub obiekty wirtualne. Dzięki zrozumieniu środowiska w ten sposób, doświadczenia AR mogą być bardziej świadome kontekstowo, zapewniając w ten sposób bardziej realistyczne i angażujące doświadczenia.

Nadzór:
Wspólna segmentacja semantyczna i klasyfikacja sceny mogą być również wykorzystywane w systemach nadzoru do automatycznej identyfikacji i śledzenia osób i obiektów w środowiskach wewnętrznych. Może to obejmować identyfikację osób, wykrywanie podejrzanych zachowań i monitorowanie ogólnej aktywności w przestrzeni. Poprzez zrozumienie środowiska w ten sposób, systemy nadzoru mogą poprawić swoją zdolność do wykrywania i reagowania na zagrożenia bezpieczeństwa.

Wnioski:
Wspólna segmentacja semantyczna i klasyfikacja sceny w środowiskach wewnętrznych jest wymagającym, ale ważnym obszarem badawczym o wielu potencjalnych zastosowaniach. Wiąże się to z wykorzystaniem zaawansowanych technik widzenia komputerowego, solidnych i wydajnych algorytmów oraz starannej oceny w rzeczywistych środowiskach wewnętrznych. W miarę rozwoju technologii, prawdopodobnie zostaną zidentyfikowane nowe przypadki użycia i zastosowania, i nadal będzie to aktywny obszar badań.

\newpage % Rozdziały zaczynamy od nowej strony
\newpage % Rozdziały zaczynamy od nowej strony.
\section{Wstep teoretyczny}

% # TODO: Zamienić obrazki definicji klasyfikacji i segmentacji na te z mojego zbioru

\subsection{Nadzorowane uczenie maszynowe}
Uczenie maszynowe to podzbiór sztucznej inteligencji, który obejmuje rozwój algorytmów i modeli statystycznych, które umożliwiają komputerom uczenie się z danych, bez wyraźnego programowania. Jest to metoda uczenia komputerów, aby rozpoznawały wzorce i dokonywały przewidywań na ich podstawie.

Uczenie nadzorowane to rodzaj uczenia maszynowego, w którym algorytm jest szkolony na etykietowanym zestawie danych, gdzie pożądane wyjście dla danego wejścia jest już znane. W kontekście głębokiego uczenia się, algorytmy uczenia nadzorowanego wykorzystują sieci neuronowe do uczenia się z danych i dokonywania przewidywań.

Jedną z głównych zalet wykorzystania głębokiego uczenia do uczenia nadzorowanego jest możliwość uczenia się złożonych i nieliniowych zależności z danych. Głębokie sieci neuronowe, z ich wieloma warstwami, mogą uczyć się i reprezentować wielowymiarowe i abstrakcyjne cechy danych, co pozwala im osiągnąć satysfakcjonujące rezultaty w wielu zadaniach. Dodatkowo, algorytmy głębokiego uczenia mogą obsługiwać duże ilości danych i mogą być łatwo zrównoleglone, co pozwala na skrócenie czasu treningu.

Istnieją jednak również ograniczenia w stosowaniu głębokiego uczenia do uczenia nadzorowanego. Jednym z ograniczeń jest konieczność posiadania dużej ilości oznaczonych danych. Aby wytrenować głęboką sieć neuronową, wymagana jest znaczna ilość oznaczonych danych, które nie zawsze mogą być łatwo dostępne lub łatwe do uzyskania. Dodatkowo, algorytmy głębokiego uczenia mogą być podatne na przepełnienie, zwłaszcza gdy ilość danych jest ograniczona. Może to prowadzić do słabej generalizacji na niewidzianych danych.
\subsection{Głębokie uczenie i konwolucje}
Uczenie głębokie odnosi się do podzbioru uczenia maszynowego, które charakteryzuje się wykorzystaniem głębokich sieci neuronowych, które składają się z wielu warstw sztucznych neuronów. W kontekście wizji komputerowej, głębokie uczenie zostało wykorzystane do osiągnięcia wielu sukcesów w szerokim zakresie zadań, w tym klasyfikacji obrazów, wykrywania obiektów i segmentacji semantycznej.

Jedną z kluczowych zalet głębokiego uczenia w wizji komputerowej jest zdolność do automatycznego uczenia się hierarchicznych reprezentacji obrazów, które mogą być wykorzystane do wyodrębnienia wysokopoziomowych cech, które są wysoce zróżnicowane dla danego zadania. Stanowi to kontrast do tradycyjnych metod widzenia komputerowego, które zazwyczaj opierają się na ręcznie opracowanych cechach, które są zaprojektowane tak, aby były informatywne dla konkretnego zadania.

Uczenie głębokie, a konkretnie głębokie konwolucyjne sieci neuronowe (CNN), zostały szeroko zaadoptowane w dziedzinie widzenia komputerowego, z wieloma sukcesami w różnych zadaniach, takich jak klasyfikacja obrazów, wykrywanie obiektów i segmentacja semantyczna. W tym rozdziale zostanie przedstawiony krótki przegląd niektórych najważniejszych kamieni milowych w rozwoju głębokich CNN dla wizji komputerowej, ze szczególnym uwzględnieniem klasyfikacji obrazów, jako zadania, którego rozwój przyczynił się do znacznego rozrostu wiedzy wsród innych zadań.

Jedną z najwcześniejszych i najbardziej wpływowych prac w dziedzinie głębokich CNN dla wizji komputerowej jest "ImageNet Classification with Deep Convolutional Neural Networks" autorstwa Alexa Krizhevsky'ego, Ilya Sutskevera i Geoffrey'a Hintona (2012). W pracy tej przedstawiono zastosowanie głębokich sieci neuronowych do klasyfikacji obrazów i osiągnięto najwyższej wyniki na zbiorze danych ImageNet. Praca ta wyznaczyła nowy punkt odniesienia dla klasyfikacji obrazów i zapoczątkowała szerokie zastosowanie CNN w zadaniach widzenia komputerowego.

W kolejnych latach wielu badaczy zaproponowało różne modyfikacje i ulepszenia podstawowej architektury CNN. Jednym z ważnych wkładów jest architektura Inception, wprowadzona przez Szegedy i in. w "Going Deeper with Convolutions" (2014). Architektura Inception wykorzystuje kombinację różnych rozmiarów filtrów konwolucyjnych do ekstrakcji cech w wielu skalach, co pozwala sieci uczyć się bardziej złożonych i abstrakcyjnych cech niż wcześniejsze architektury.

Kolejną kluczową innowacją w rozwoju głębokich CNN dla wizji komputerowej jest wykorzystanie połączeń rezydualnych, które zostało zaproponowane przez He i in. w "Deep Residual Learning for Image Recognition" (2016). Połączenia rezydualne pozwalają na trenowanie bardzo głębokich sieci poprzez ułatwienie optymalizacji gradientów i zapobieganie problemowi znikającego gradientu. Architektura ResNet, która wykorzystuje połączenia rezydualne, wykazała, że osiąga lepszą wydajność w zadaniu klasyfikacji ImageNet niż poprzednie architektury.

Podsumowując, głębokie CNN są wysoce efektywne w zadaniach widzenia komputerowego, takich jak klasyfikacja obrazów. Rozwój głębokich CNN zaznaczył się kilkoma ważnymi kamieniami milowymi, w tym wykorzystaniem głębokich architektur, różnych architektur, takich jak Inception, oraz wykorzystaniem połączeń rezydualnych. Te innowacje doprowadziły do znacznej poprawy wydajności na zbiorze danych ImageNet i zainspirowały dalsze badania w innych zadaniach widzenia komputerowego.
\subsection{Segmentacja semantyczna}
Segmentacja semantyczna jest zadaniem w wizji komputerowej, które ma na celu przypisanie semantycznej etykiety do każdego piksela w obrazie. Zadanie to ma wiele praktycznych zastosowań, takich jak rozumienie sceny, wykrywanie obiektów i edycja obrazów. W tym rozdziale przedstawimy przegląd niektórych najważniejszych kamieni milowych w rozwoju głębokich splotowych sieci neuronowych (CNN) do segmentacji semantycznej, analizując kluczowe prace w tej dziedzinie.

Jednym z najwcześniejszych i najbardziej wpływowych artykułów w dziedzinie głębokich CNN do segmentacji semantycznej jest "Fully Convolutional Networks for Semantic Segmentation" autorstwa Longa, Shelhamera i Darrella (2015)\cite{fcn}. W pracy tej, zaprezentowanej na konferencji Computer Vision and Pattern Recognition (CVPR), przedstawiono architekturę sieci w pełni splotową (FCN) do segmentacji semantycznej. Architektura FCN wykorzystuje serię warstw konwolucyjnych i upsamplingu do produkcji gęstych predykcji per-piksel. Praca ta pokazała, że CNN mogą być wykorzystane do predykcji na poziomie pikseli i stworzyła podstawy dla wielu późniejszych podejść do segmentacji semantycznej.

Innym kluczowym wkładem w dziedzinie segmentacji semantycznej jest "U-Net: Convolutional Networks for Biomedical Image Segmentation" autorstwa Ronneberger, Fischer i Brox (2015)\cite{ronneberger2015u}. W pracy tej, zaprezentowanej na międzynarodowej konferencji Medical Image Computing and Computer-Assisted Intervention (MICCAI), przedstawiono architekturę U-Net do segmentacji obrazów biomedycznych. Architektura U-Net wykorzystuje kombinację warstw konwolucyjnych i poolingowych do ekstrakcji cech w wielu skalach oraz serię warstw upsamplingu do produkcji gęstych predykcji per-pikselowych. Praca ta pokazała, że architektura U-Net dzięki zastosowaniu połączeń pomijających (skipping connections) jest w stanie znacznie lepiej rekonstruować obraz. Szczególnie dotyczy to elementów małej skali, które wcześniej były pomijane przez FCN. Praca ta została szeroko wykorzystana w obrazowaniu medycznym i nie tylko.

Kolejną ważną pracą w dziedzinie segmentacji semantycznej jest "DeepLab: Semantic Image Segmentation with Deep Convolutional Nets, Atrous Convolution, and Fully Connected CRFs" autorstwa Chen, Papandreou, Kokkinos, Murphy i Yuille (2016)\cite{deeplab}. W pracy tej, zaprezentowanej na International Conference on Computer Vision (ICCV), przedstawiono architekturę DeepLab do segmentacji semantycznej. Architektura DeepLab wykorzystuje rozszerzony splot (atrous convolution) do zwiększenia pola widzenia warstw konwolucyjnych oraz warunkowe pola losowe (CRF) do dopracowania predykcji. Praca ta pokazała, że użycie rozszerzonego splotu i CRF może poprawić efekty segmentacji semantycznej.

Podsumowując, segmentacja semantyczna jest zadaniem o dużym znaczeniu w wizji komputerowej, a głębokie CNN okazały się wysoce skuteczne w rozwiązywaniu tego zadania. Rozwój głębokich CNN do segmentacji semantycznej został oznaczony przez kilka ważnych kamieni milowych, w tym wprowadzenie FCN przez Long et al, U-Net przez Ronneberger et al i DeepLab przez Chen et al. Te architektury wyznaczyły nowe standardy w segmentacji semantycznej i zostały szeroko przyjęte w różnych dziedzinach zastosowań.
\subsection{Definicje zadań}
\subsubsection{Klasyfikacja sceny}
\begin{figure}[ht!]
    \includegraphics[width=\textwidth]{img/scene_class.png}
    \caption{Problem różnorodności wewnątrzklasowej oraz wieloznaczności semantycznej \cite{zeng2021deep}.}
    \label{fig:scene-class}
\end{figure}

Zadanie klasyfikacji sceny polega na przyporządkowaniu kategorii miejsca, w które przedstawia obraz. Istnieje duża różnica między klasyfikacja obrazka a klasyfikacją sceny. Klasyfikacja obrazka jako taka zajmuje się przyporządkowaniem klasy obiektu pierwszoplanowego, np. czy na obrazie znajduje się pies, czy kot. Klasyfikacja sceny natomiast musi wziąć pod uwagę wszystkie cechy obrazu, zarówno tła, jak i pierwszego planu, by określić odpowiednie miejsce. 

W kontekście środowisk wewnętrznych, klasyfikacja scen stanowi wyzwanie ze względu na zmienność scen wewnętrznych, obecność okluzji oraz fakt, że ten sam typ sceny może wyglądać inaczej na różnych obrazach. Wyróżniamy między innymi problem różnorodności wewnątrz klasowej oraz wieloznaczności semantycznej, co zostało przedstawione na rys. \ref{fig:scene-class}. Pierwszy z nich polega na fakcie, iż jedno miejsce może zostać przedstawione w bardzo różnej konfiguracji m.in. oświetlenia, ekspozycji, obiektów znajdujących się na obrazie. Drugi jest związany z występowaniem tych samych obiektów dla różnych klas scen.

\subsubsection{Segmentacja obrazu}
\begin{figure}[ht!]
    \includegraphics[width=\textwidth]{img/segment.png}
    \caption{Segmentacja wewnątrz pomieszczeń \cite{zhang2018context}.}
    \label{fig:segment}
  \end{figure}
  
Zadanie segmentacji obrazu to przyporządkowanie każdemu pikselowi etykiety takiej jak ,,łóżko'', ,,kanapa'' lub ,,umywalka'', do każdego piksela w obrazie (rys. \ref{fig:segment}). W rezultacie obraz zostaje podzielony na homogeniczne regiony pod względem pewnych własności. Segmentacja może być reprezentowana jako tablica 2D, gdzie każdy element odpowiada pikselowi w obrazie wejściowym i ma wartość wskazującą jego etykietę klasy.
  
Zadanie segmentacji można rozszerzyć do zadania segmentacji instancji (ang. instance segmentation), czyli segmentacji klasycznej rozszerzonej o rożróżnienie poszczególnych obiektów w ramach tej samej klasy. W przypadku klasycznej wersji nie jesteśmy w stanie rozróżnić dwóch stojących obok siebie łóżek, gdyż mapa segmentacji jest dla nich jednakowa. Segmentacja instacji pozwala natomiast takie rozróznienie uczynić. Segmentacja semtantyczna w dalszej części pracy będzie odnosić się do klasycznej wersji. Segmentacja instancji nie jest tematem pracy.
\subsection{Uczenie wielozadaniowe}
% # TODO
\newpage % Rozdziały zaczynamy od nowej strony
\newpage % Rozdziały zaczynamy od nowej strony.
\section{Rozwiązanie}
Zadania wizji komputerowej moga zostać podjęta na wiele sposobów. Szczególnie iteresujące są podjeścia do zadania łącznej segmentacji semantycznej i klasyfikacji sceny we wnętrzach. W tym rozdziale przedstawione zostaną wybrane metody, które zostały sprawdzone w ramach analizy problemu. W swoich rozważaniach będę bezpośrednio odnosił się do pytań badawczych postawionych w celu pracy, a więc: 

\begin{itemize}
    \item Jak można zaprojektować model oparty na głębokim uczeniu do wspólnej segmentacji semantycznej i klasyfikacji scen w środowiskach wewnętrznych?
    \item Czy przestrzeń reprezentacji po wytrenowaniu na zadaniu segmentacji semantycznej może być użyta do zadania klasyfikacji sceny?
    \item Jak dobrze proponowany model radzi sobie na dużym zbiorze danych scen wewnętrznych i jak wypada w porównaniu z aktualnymi metodami segmentacji semantycznej i klasyfikacji scen osobno?
    \item Jak proponowany model może być wykorzystany do poprawy wydajności w robotyce mobilnej?
\end{itemize}
Opis rozwiązań problemu zostanie poprzedzony przeglądem rozwiązań. Analiza dotychczasowych pozwoli lepiej ukierunkować badania. Korzystając z doświadczenia innych, będzie można wyrobić sobie intuicję, która pomoże podejmować konkretne decyzje.

\subsection{Przegląd rozwiązań}
Przegląd literatury jest kluczowym aspektem każdej pracy naukowej. W tym rozdziale zostaną przedstawione wyłącznie rozwiązania obejmujące łączną segmentację semantyczną oraz klasyfikację sceny. Szczególny nacisk położony zostanie na architektury głębokich sieci neuronowych z dogłębną analizą przepływu inferencji przez nie.
\vspace{0.5cm}
Niestety przyjetę założenia w pracy nie zostały opisane przez nikogo wczesniej, zgodnie z najlepszą wiedzą autora. Niektóre prace naukowe przedstawiają ten sam problem to jest klasyfikacji i segmentacji łącznie, ale obejmują go w innej domenie danych. Z drugiej artykuły obejmujące środwiska wnętrz są dobrze zdefiniowane, jednak często w swoich rozwiązaniach korzystają z obrazu głębki, który nie zawiera się w zakresie badań tej pracy. Nie mniej wszystkie poniższe artykuły stanowią cenne źródło informacji, które należy mniej lub bardziej dostoswać do rozważanego problemu.

Pierwszym z prezentowanych artykułów jest ,,Describing the Scene as a Whole: Joint Object Detection, Scene Classification and Semantic Segmentation'' autorstwa Yao j. et al (2012)\cite{yao2012describing}. Prezentuje on algorytm, który ówcześnie wyznaczył najlepsze podejście (ang. state-of-the-art (SOTA)). Autorzy wskajzują tutaj, że połączenie rozważanych zadań okazało się owocne nie tylko pod względem jakości, ale również wydajności w kontekście czasowym. Yao J. et al zwracają uwagę na połączenie szeregowe, które niestety propaguje błąd w kolejnych zadaniach, a było dotychczasowo szeroko stosowane. W swojej pracy wykorzysztują podejście równolegle zgodne z rysunkiem \ref{fig:scene-as-a-whole}. Podsumowujac, ,,Describing the Scene as a Whole: Joint Object Detection, Scene Classification and Semantic Segmentation'' nie jest propozycją architektury głębokiej sieci. Wskazuje on na problemy z łączenie, zadań szeregowo, jednocześnie udowadniając, że taka praktyka był ówcześnie stosowana, więc nie można uznawać stosowania połączenia szeregowego jako niedopuszczalnego.

\begin{figure}[ht!]
    \includegraphics[width=\textwidth]{img/joint-segmentation-and-classification.png}
    \caption{Describing the Scene as a Whole: Joint Object Detection, Scene Classification and Semantic Segmentation (2012) \cite{yao2012describing}.}
    \label{fig:scene-as-a-whole}
\end{figure}

Artykuł ,,Let there be Color!: Joint End-to-end Learning of Global and Local Image Priors for Automatic Image Colorization with Simultaneous Classification 2016 \cite{iizuka2016let}.'' (2016) przedstawia rozwiązanie problemu jednoczesnego klasyfikowania sceny oraz kolorowania zdjęć. Do realizacji zadania kolorowania potrzebna jest sematyczna maska. Wynika z tego, że kolorowanie jest rozszerzeniem segmentacji semantyzcnej. Rozumiejąc towarzyszczące analogie można przejść do analizy rozwiązania. Przedstawiona architektóra (rys.\ref{fig:parrarel-arch}) jest przykładem sieci wielozadaniowej, uzywającej miękkiego dzielenia parametrów, ale tylko i wyłącznie w obrębie pierwszej części sieci. Szczególnie ciekawa jest konkatencja cech wysokiego poziomu (Global Features Network) z cechami średniopoziomowymi (Mid-Level Features Network), która ma miejsce w warstwie fuzji (Fusion layer). Iizuka et al. formułują wniosek oznajmiający o kluczowym znaczeniu tej warstwy w kontekście całego zadania. Wiedza o scenie zdjęcia może dostarczyć informacji wpływających na decyzję, czy na obrazie znajduje się niebo czy trawa. Rozważając sceny wnętrz oczywiste jest, że nie będzie tam takich grup semantycznych. Podsumowując, cechy nauczone na zadaniach klasyfikacj i segmentacji, mogą wzjamniej pozytwnie na siebie wpływać, realizując pozytywny transfer.

\begin{figure}[ht!]
    \includegraphics[width=\textwidth]{global-local-features.png}
    \caption{Let there be Color!: Joint End-to-end Learning of Global and Local Image Priors for Automatic Image Colorization with Simultaneous Classification 2016 \cite{iizuka2016let}.}
    \label{fig:parrarel-arch}
\end{figure}

Zastosowanie łącznej segmentacji oraz klasyfikacji tym razem w domenie medycznej przedstawia ,,Y-Net: Joint Segmentation and Classification for Diagnosis of Breast Biopsy Images'' \cite{mehta2018net} (2018). Zadanie te są realizowane przez twarde dzielenie parametrów w kontekście uczenia wielozadaniowego (rys.\ref{fig:y-net}). Architektura jest prostym rozszerzeniem klasycznego U-Netu. Autorzy wskazują, że taki zabieg powodują dużą modulatność, ponieważ do dowolnego modelu segmentacji można podłączyć sieć klasyfikacyjną. Przeprowadzone eksperymenty dla segmentacji udowodniły, że dokładność pozostała na tym samym poziomie. W przypadku klasyfikacji wyniki były wyższe niż dotychczasowe SOTA na tym zbiorze. Jako funkcję straty autorzy użyli sumę entropii skrośnej każdego z zadań. Podsumowując zadanie zadanie segmentacji osiągneło ten sam wysoki wynik co SOTA, a zadanie klasyfikacji ustanowiło nowe SOTA na tym zbiorze uczać się znacznie mniej parametrów.
\begin{figure}[ht!]
    \includegraphics[width=\textwidth]{y-net-new.png}
    \caption{Y-Net: Joint Segmentation and Classification for Diagnosis of Breast Biopsy Images 2018 \cite{mehta2018net}.}
    \label{fig:y-net}
\end{figure}


Najbliższy artykuł tej pracy inżynierskiej jest ,,Efficient Multi-Task RGB-D Scene Analysis for Indoor Environments'' \cite{9892852} (2022), który został opublikowany w czasie tworzenia tej pracy. Przedstawia on jedną głęboką sieć neuronową rozwiązującą następujące zadania: segmentacja semnatyczna oraz segmentacja instacji (łącznie ang. pantopic segmentation), estymacja orientacji instacji oraz klasyfikacja sceny. Rozważaną przez autorów domeną są podbnie jak w przypadku tej pracy sceny wnętrz. Znaczą róznicą poza dodatkowymi zadaniami jest użycie przez zespołu Seichter et al. informacji o głębi. Zgdonie z wnioskami z nieniejszego artykułu przetwarzanie łączne obrazów RGB i głębi jest kluczowe z punktu widzenia jakości predykcji. Każde z zadań zostało na początku trenowane osobno by ustalić punkt odniesienia. Architektura jest przedstawiona na rysunku \ref{fig:emsanet}. Autorzy zdecydowali się na twarde dzielenie parametrów, argumentując całkowitą niezależnością w przypadku chęci wyłączenia jednego zadań z wnioskowania. Trening każdej sieci z osobna był rozważany pod względem wielu backbone'ów ze zróżnicowaniem na uczenie wyłącznie obrazu głebi, obrazu RGB lub RGB-D. Generalnie w przypakdu segmentacji oraz klasyfikacji większy backbone wpływał na polepszenie wyników. Trenując zadania łacznie zdecydowano się na ważoną sumę entropii skrośnej dla zadania segmentacji i klasyfikacji w proporcjach odpowiednio 3:1. Przyjęty krok uczenia, będąć sprawdzonym przez przeszukiwanie liniowe (ang. grid search), jest wyjątkowo duży, bo wynosi 0.02. Autorzy zastosowali zaawansowane techniki dostosowywania kroku czenia w trakcie treningu poprzez użycie planista polityki jednego cyklu (ang. one cycle policy scheduler). Jako optymalizator użyto SGD z momentem oraz drobną regularyzacją. Podsumowujac, zgodnie z prezentowanymi wynikami na wspólnej segmentacji oraz klasyfikacji autorom nie udało się polepszyć działania modelu na segmentacji semantycznej. Z powodzeniem jedank wzrosła dokładność klasyfikacji na zbiorze NYUv2.

\begin{figure}[ht!]
    \centering
    \includegraphics[width=\textwidth]{emsanet.png}
    \caption{Efficient Multi-Task RGB-D Scene Analysis for Indoor Environments \cite{9892852}}
    \label{fig:emsanet}
\end{figure}


\subsection{Rozwiązanie problemu}
W tym rozdziale zostaną przedstawione eksperymenty, które wykonanano w celu zbadania uczenia wielozadniowego segmentacji semantycznej oraz klasyfikacji sceny w domenie pomieszczeń. Pierwszym założeniem jakiego dokonano było wyznaczenie punktu odniesienia. Z punktu widzenia pracy nałatwiej byłoby znaleźć gotowe wyniki segmetnacji oraz klasyfikacji sceny na wybranym zbiorze danych. Niestety żadne z przytaczanych rozwiązań nie odpowiada w pełni zakresowi pracy. Postanowiono stworzyć taki punkt odniesienia samemu przez analogiczne trenowanie sieci segmentacyjnej oraz klasyfikacyjnej osobno.

Posiadając taką wiedzę eksperymentowano dalej z róznymi architekturami uczenia wielozadaniowego. Wybrano uczenie łączne o twardym dzieleniu parametrów. Podejście to ma wiele zalet. \cite{mehta2018net} podkreśla łatwość i wszechstronność implementaji. Wystarczy dołączyć do modelu część klasyfikacyjną. Co więcej wszyscy autorzy (\cite{mehta2018net}, \cite{9892852}) chwalą znacznie mniejszą ilość parametrów sieci co bezpośrednio wpływa na czas treningu oraz wnioskowania. Archtektura sieci przedstawia sie następująco\ref{fig:multitask}. Jest to DeepLabv3 rozszerzony za enkoderem o sieć klasyfikacyjną podbnie jak w artykule \cite{mehta2018net}, gdzie rozszerzono sieć U-Net.
\begin{figure}[ht!]
    \centering
    \includegraphics[width=0.25\textwidth]{no-image.png}
    \caption{Architektura wielozadaniowej sieci.}
    \label{fig:multitask}
\end{figure}

\subsubsection{Uczenie wielozadaniowe}
Uczenie wielozadaniowe zostało zrealizowane przez architekturę z rysunku \ref{fig:multitask}. Trening polegał na aktualizowaniu wag całego dostępnego modelu zgodnie zgodnie z propagają wsteczną zagregowanej funcji straty $\lambda$. Zaimplementowano ją jako sumę funkcji strat na każdym z zadań, tak jak w przypadku \cite{mehta2018net}. Nie stosowano ważenia zadań \cite{9892852}.
\begin{equation*}
    \lambda = \lambda_{segmentacja} + \lambda_{klasyfikacja}
\end{equation*}

\subsubsection{Wyłącznie klasyfikacja}
W celu określenia punktu odniesienia wytrenowano model zapominając o podsieci do wyznaczania segmetnacji semantycznej. Technicznie skorzystano z modelu wielozadaniowego. Parametry modułów arhitektury takie jak dekoder oraz głowa segmentacyjna zostały zamrożone oraz nie zostały podawane optymalizatorowi w trakcie treningu. Funckja straty $\lambda$ została ograniczona wyłącznie do straty na klasyfikacji poprzez wyzerowanie w kazdym kroku straty na segmentacji

\begin{gather*}
    \lambda = \lambda_{segmentacja} + \lambda_{klasyfikacja} \\
    \lambda_{segmentacja} = 0
\end{gather*}
\subsubsection{Wyłącznie segmentacja}
Analogicznie jak w przypadku klasyfikacji należało określić punkt odniesienia również w przypadku segmentacji. Procedura była taka sama jak w przypadku klasyfikacji. Model wielozadaniowy zamrożono w części klasyfikacyjnej oraz wyłączono zamrożone parametry z optymalizacji. Funkcja straty $\lambda$ została przedstawiona jako
\begin{gather*}
    \lambda = \lambda_{segmentacja} + \lambda_{klasyfikacja} \\
    \lambda_{klasyfikacja} = 0
\end{gather*}
\subsubsection{Finetuning}
Znaną techniką transferu wiedzy jest finetuning. W tym przypadku skorzystano z wytrenowanego enkodera ResNet wytrenowanego na dużej bazie ImageNet. Uczenie przebiegało w dwóch fazach. W pierwszej zamrożono enkoder i starano się osiągnać jak najlepsze rezultaty dysponukąć podsieciami klasyfikacyjną i segmetacyjną. Wynika z tego, że pierwszy etap to ni innego jak uczenie wielozadaniowe ale z wyłączonym enkoderem. Dopiero w drugim etapie odmrażany jest również enkoder. Sytuacja wtedy przypomina wcześniej omawiane uczenie wielozadaniowe. Jendakże, kluczowy jest dobór hiperparametrów. W pierwszym etapie uczenie przebiega z pewny krokiem uczenia. W drugim zaś krok uczenia jest znacznie mniejszy.
\begin{equation*}
    \lambda = \lambda_{segmentacja} + \lambda_{klasyfikacja}
\end{equation*}
\subsubsection{Równoległa klasyfikacja z segmentacji}
Podejście transferu wiedzy można lekko zmodyfikować. Skorzystano z wcześniej przygotowanych wag będących wynikiem wcześniej wspomnianej wyłącznej segmetnacji. Zamrożono enkoder oraz podsieć segmentacyjną oraz wyłączono te parametry z optymalizacji. Następnie dysponując samą podsiecią klasyfikacyjną przeprowadzono trening. Funckja straty była następująca:
\begin{gather*}
    \lambda = \lambda_{segmentacja} + \lambda_{klasyfikacja} \\
    \lambda_{segmentacja} = 0
\end{gather*}
\subsubsection{Szeregowa klasyfikacja z segmentacji}.
Rozwiązaniem odbiegającym od reszty jest przeprowadzenie szeregowej klasyfikacji z segmentacji. Architektura przedstawia się zgodnie z rysunkiem \ref{fig:multitask-parrarel}. W tym eksperymencie sprawdzono jak można skorzystać zgotowych predykcji dotyczących segmentacji. Model aż do głowy segmentacyjnej włącznie został zamkrożony oraz wyłączony z optymalizacji. Zmieniają się tylko wagi części klasyfikacyjnej.

\begin{equation*}
    \lambda = \lambda_{segmentacja} + \lambda_{klasyfikacja}
\end{equation*}







\begin{figure}[ht!]
    \centering
    \includegraphics[width=0.25\textwidth]{no-image.png}
    \caption{Arhitekrura sieci szeregowej.}
    \label{fig:multitask-parrarel}
\end{figure}

W celu łatwej oraz dokładniej ewaluaji 

W celu lepsze ewaluacji uczenia wielozadaniowego zdecydowano uściślij wszystkie parametry sieci takie jak architekura jest ta sama zeby dało sie porownac nie

nie ma sensu wszystkich scenariusz bo tak

opisać że nie ma sensu porównywać 



Poza uczeniem łącznym zbadano też inne znane techniki uczenia jak finetuning.


% % * multitask Learning
% % * jednozadaniowe
% % * wielozadaniowe
% W celu realizacji zadania zdecydowano się na architekturę (najbliższą Y-Netu) o wspólnym enkoderze i o osobnych głowach, służących do egzekwowania konkretnych zadań (rys. \ref{fig:cep_arch}). Decyzja podyktowana była względnie prostą implementacją rozszerzenia wielu modeli segmentacji semantycznej o dodatkową głowę klasyfikacyjną. Co więcej stwierdzono, że ograniczenie się tylko do jednego backbone'u jest niesłychanie korzystne, gdyż znacząco ogranicza ilość parametrów sieci, co bezpośrednio przekłada się m.in. na czas inferencji. Należy zwrócić uwagę na fakt, iż właściwie zdecydowana większość parametrów znajduje się własnie w enkoderze.

% Mając na uwadze, że symultaniczne uczenie może negatywnie wpływać na jakość uczenia obu zadań, eksperymenty przeprowadzono etapowo. Pierwszym etapem było uczenie jednozadaniowe. Eksperymenty polegały na sprawdzeniu jakości segmentacji oraz klasyfikacji osobno. Wykorzystano do tego tę samą archtekturę, która używana była poźniej w drugim etapie. Mianowicie, mając dwie głowy każdorazowo zamrażano głowę nie biorącą udziału w uczeniu (rys. \ref{fig:arch-scene-seg}). Zapewnia to pewność posiadania tej samej architektury, a w szczególności rzetelne porównanie z etapem uczenia wielozadaniowego.

% Drugim etapem było przeprowadzenie eksperymentów w uczeniu wielozadaniowym (rys. \ref{fig:arch-full}). Funkcja celu zdefiniowana była jako suma wartości funkcji celów dla obu zadań. W wyniku progpagacji wstecznej wagi aktualizowane były zgodnie z zagregowaną stratą.

% Ostatecznie porównano jakość na przesztreni obu etapów.

% \begin{figure}[ht!]
%     \centering
%     \includegraphics[width=0.75\textwidth]{cep_arch.png}
%     \caption{Architektura sieci zastosowana w pracy inżynierskiej.}
%     \label{fig:cep_arch}
% \end{figure}

% \begin{figure}[ht!]
%     \centering
%     \begin{subfigure}[b]{0.49\textwidth}
%         \centering
%         \includegraphics[width=\textwidth]{arch:scene.png}
%         \caption{Architektura sieci wyłącznie w zadaniu klasyfikacji.}
%     \end{subfigure}
%     \hfill
%     \begin{subfigure}[b]{0.49\textwidth}
%         \centering
%         \includegraphics[width=\textwidth]{arch:seg.png}
%         \caption{Architektura sieci wyłącznie w zadaniu segmentacji semantycznej.}
%     \end{subfigure}
%     \caption[]{Podejście jednozadaniowe.}
%     \label{fig:arch-scene-seg}
% \end{figure}

% \begin{figure}[ht!]
%     \centering
%     \includegraphics[width=0.75\textwidth]{arch:full.png}
%     \caption{Architektura sieci jako uczenie wielozadaniowego.}
%     \label{fig:arch-full}
% \end{figure}
\newpage % Rozdziały zaczynamy od nowej strony
\clearpage
\newpage % Rozdziały zaczynamy od nowej strony.

\section{Wyniki}

W tym rozdziale zostaną przedstawione empiryczne wyniki badań nad wspólną segmentacją semantyczną i klasyfikacją sceny w środowiskach wewnętrznych. Badania mają na celu opracowanie i ocenę różnych znanych i aktualnych technik uczenia głębokich sieci neuronowych. Aby to osiągnąć, przeprowadzono serię eksperymentów na zbiorze na reprezentacyjnych zbiorach danych. Analiza dotyczyła zarówno miar jakości sensu stricto, jak i miar wydajnościowych proponowanych metod. Rozważono różne metryki oceny, takie jak ogólna dokładność, indeks Jaccarda znany w literaturze jako intersection over union (IoU), miara F1 i wydajność obliczeniowa. Wyniki uzyskane w tym rozdziale zapewnią cenny wgląd w mocne strony i ograniczenia proponowanych metod.

\subsection{Analiza miar jakości}
W pierwszej kolejności metody zostaną zbadane pod względem wymienionych wcześniej miar jakości w postaci ogólnej — niezagregowanej, osobno dla segmentacji oraz klasyfikacji. Omawiane metryki należy rozumieć jako średnia miara jakości na każdej z~klas, a więc makrośrednie. Makrośrednie metryki są stosowane przy ocenie wydajności algorytmów, ponieważ zapewniają bardziej wszechstronną ocenę ogólnej jakości algorytmu. Metryki makrośrednie uwzględniają wydajność algorytmu na wszystkich klasach obiektów i regionów w obrębie sceny, a nie tylko koncentrują się na jakości na najbardziej powszechnych lub najłatwiejszych do sklasyfikowania klasach. W przypadku stosowania metryki makrośredniej, jakość dla każdej klasy jest obliczana oddzielnie, a ogólna jakość jest obliczana jako średnia jakości poszczególnych klas. Stanowi to kontrast do metryki mikrośredniej, która oblicza ogólną jakość poprzez zsumowanie całkowitej liczby wyników dla wszystkich klas.
Użycie makrośrednich metryk może być szczególnie ważne w scenariuszach, w których liczba instancji każdej klasy jest niezrównoważona lub gdy istnieje duża liczba klas. W takich przypadkach mikrośrednie metryki mogą być mylące, ponieważ mogą być pod silnym wpływem najbardziej powszechnych klas, podczas gdy zaniedbują te mniej powszechne. Zatem makro analiza pokaże generalne rezultaty oraz otworzy dyskusję do dalszych, bardziej pogłębionych badań na rozważanym problemem.

\vspace{0.5cm}
Rozpoczynając od segmentacji, rozważamy 3 scenariusze testowe. Pierwszym z nich jest uczenie wyłącznie klasyfikacji rozumianej jako uczenie enkodera i sieci segmentacyjnej z pominięciem cześć klasyfikacyjnej. Pozwoli to odpowiedzieć na pytanie, czy bardziej zaawansowane techniki uczenia polepszą, a może pogorszą działanie modelu. Drugim scenariuszem jest uczenie wielozadaniowe, gdzie cały model jest odmrożony, a błąd jest propagowany zarówno przez segmentację, jak i klasyfikację. Ostatnim eksperymentem jest sprawdzenie technik transferu wiedzy, a szczególne tak zwanego finetuningu. Model w~pierwszym etapie uczy się przy zamożnym enkoderze, dopiero na koniec jest odmrażany w celu dostrojenia wyników.

\begin{figure}[ht!]
    \centering
    \includegraphics[width=\textwidth]{img/pl-res/Segmentacja-semantyczna.jpeg}
    \caption{Porównanie miar IoU oraz dokładnosci dla segmentacji sceny.}
    \label{fig:macro-segmentation}
\end{figure}

Analizując rysunek \ref{fig:macro-segmentation} nie trudno zauważyć, że najlepsze rezultaty otrzymano w dla uczenia wyłącznie segmentacji. Kolejnym wynikiem jest uczenie wielozadaniowe. Jako najsłabsze podejście okazuje się metoda finetunowania. Relacje jakości są zachowane dla każdej z metryk, a więc zarówno dla miary IoU, jak i zbilansowanej dokładności (bAcc). Widać, że miara IoU wypada gorzej niż bAcc. Wyniki mogą sugerować, że trudno jest przeprowadzić transfer wiedzy z ImageNetu, gdyż finetuning wypada najsłabiej. Jest to najprawdopodobniej spowodowane zupełnie innym rozkładem klas na wspomnianej bazie. Analiza sceny w przeciwieństwie do klasyfikacji najczęściej cechuje się długoogonowym rozkładem klas. Drugim istotnym szczegółem jest fakt, iż wagi dekodera i głowy segmentacyjnej są losowe. Uczenie wielozadaniowe zgodnie z zakładanymi wynikami nie polepsza segmentacji, gdyż łączna przestrzeń segmentacji i klasyfikacji jest niewątpliwie trudniejsza do optymalizacji.

\vspace{0.5cm}
Przechodząc do klasyfikacji, wyróżniamy 5 scenariuszy testowych. Pierwszym jest uczenie wyłącznie klasyfikacji, analogicznie jak wyżej, a więc przy wyłączonej części segmentacyjnej. Kolejnymi są wspomniane wcześniej uczenie wielozadaniowe oraz finetuning. Do nowych scenariuszy zaliczamy bezpośrednią oraz pośrednią klasyfikację z~segmentacji.
\begin{figure}[ht!]
    \centering
    \includegraphics[width=\textwidth]{img/pl-res/Klasyfikacja-sceny.jpeg}
    \caption{Porównanie miar F1 oraz dokładnosci dla klasyfikacji sceny.}
    \label{fig:macro-classification}
\end{figure}

Rezultaty przedstawia rysunek \ref{fig:macro-classification}. Od razu da się zauważyć, że wyniki cechuje mniejsze odchylenie standardowe oraz, analizując łącznie miarę F1 oraz zbalansowaną dokładność, średnia. Fakt ten jest prawdopodobnie wynikiem znacznie mniejszej ilości parametrów uczących. Najlepszy rezultat uzyskuje uczenie wielozadaniowe. Ciekawym wydaje się fakt, że uczenie wyłącznie klasyfikacji jest słabsze w tym przypadku. Prawdopodobnie poprzez uczenie wielozadaniowe enkoder wygenerował lepszą przestrzeń reprezentacji, co bezpośrednio wpływa na klasyfikację sceny. Najgorszym przypadkiem jest uczenie klasyfikacji bezpośrednio z segmentacji. Nie jest to dziwne, gdyż w tym przypadku klasyfikator korzystał z zaledwie 13 kanałów.
\vspace{0.5cm}

Analizowanie jakości algorytmu dla każdej z klas osobno jest ważne, ponieważ pozwala na bardziej szczegółowe zrozumienie mocnych i słabych stron algorytmu. Rozważając ogólną jakość algorytmu przy użyciu metryki makrośredniej, nie jest od razu jasne, w~których klasach algorytm radzi sobie dobrze, a w których źle. Analizując jakość każdej klasy osobno, można zidentyfikować konkretne klasy, z którymi algorytm ma problemy i~podjąć kroki w celu poprawy wydajności w tych klasach.

\vspace{0.5cm}
Rysunek \ref{fig:classification-accuracy} przedstawia dokładność dla każdej z klas dla zadania klasyfikacji sceny. Trudno jednoznacznie określić, która z metod sprawdza się tutaj najlepiej. Uczenie wielozadaniowe wypada najlepiej dla klas: łazienka, sypialnia, salon, biuro. Uczenie wyłącznej klasyfikacji jest najlepsze dla klas jadalnia oraz kuchnia. W pozostałych przypadkach klasa inne pomieszczenia jest najlepiej wykrywana przez scenariusz finetunowania. Uczenie klasyfikacji z segmentacji nigdy nie osiąga najlepszego wyniku.
\begin{figure}[ht!]
    \centering
    \includegraphics[width=\textwidth]{img/pl-res/Dokladnosc-Klasyfikacja-Sceny.jpeg}
    \caption{Porównanie dokładności klasyfikacji sceny z rozróżniem konkretnych klas.}
    \label{fig:classification-accuracy}
\end{figure}
Biorąc pod uwagę miarę F1 (rys. \ref{fig:classification-f1}) również nie jesteśmy w stanie wyróżnić faworyzowanej metody. W porównaniu z wcześniej analizowaną dokładnością widać, że uczenie wielozadaniowe utrzymuje w~większości przypadków bardzo dobre rezultaty. Widać też, że wyniki w obrębie każdej z~klas mało różnią się między sobą.
\begin{figure}[ht!]
    \centering
    \includegraphics[width=\textwidth]{img/pl-res/Miara-F1-Klasyfikacja-Sceny.jpeg}
    \caption{Porównanie miary F1 dla klasyfikacji sceny z rozróżniem konkretnych klas.}
    \label{fig:classification-f1}
\end{figure}

\vspace{0.5cm}
Analizując rysunek \ref{fig:segmentation-acc} przedstawiający dokładność w zadaniu segmentacji semantycznej, widać, że niektóre z zadań wypadają znacznie gorzej niż pozostałe. Sytuacja ta dotyczy klas meble, stoły, obiekty. Uczenie wyłącznie segmentacji okazało się najlepsze dla klas łóżko, podłoga, meble, obiekty, obraz, tv, ściana oraz okno. Stanowi to ponad połowę wszystkich możliwych klas. Uczenie wielozadaniowe uzyskało najlepsze wyniki dla klas książki, sufit, sofa. Przypadek funetunowania nigdy nie osiągnął najlepszego rezultatu.
\begin{figure}[ht!]
    \centering
    \includegraphics[width=\textwidth]{img/pl-res/Dokladnosc-Segmentacja-Semantyczna.jpeg}
    \caption{Porównanie dokładności segmentacji z rozróżniem konkretnych klas.}
    \label{fig:segmentation-acc}
\end{figure}

Na rysunku \ref{fig:segmentation-iou} przedstawiono IoU dla segmentacji semantycznej. Widać tutaj dużą dysproporcję między klasami podłoga, ściana, a pozostałymi klasami. Jest to zrozumiałe, klasy te występują stosunkowo często na obrazie. Uczenie wyłącznie segmentacji uzyskuje najlepsze wyniki na wszystkich klasach z wyłączeniem książek oraz telewizorów. W tych przypadkach najlepsze okazuje się uczenie wielozdaniowe.

\begin{figure}[ht!]
    \centering
    \includegraphics[width=\textwidth]{img/pl-res/IoU-Segmentacja-Semantyczna.jpeg}
    \caption{Porównanie miary IoU segmentacji z rozróżniem konkretnych klas.}
    \label{fig:segmentation-iou}
    
\end{figure}
\subsection{Analiza czasowa}
Ostatnio coraz częściej mówi się o zapotrzebowaniu na zasoby sprzętowe podczas uczenia maszynowego. Głębokie sieci neuronowe, a szczególnie te przetwarzające obrazy wymagają coraz więcej zasobów obliczeniowych do prawidłowego działania. Wynika to z dwóch głównych czynników. Po pierwsze duże modele wizji komputerowej posiadają miliony parametrów. Drugim powodem jest przetwarzanie wielu obrazów, które de facto są zbiorem macierzy. Wiedzie to do większego zainteresowania zużywanymi zasobami podczas treningu oraz wnioskowania. W tym podrozdziale przedstawiona zostanie analiza czasu treningu oraz wnioskowania.

Analizując czas uczenia w przypadku kolejnych metod, odkrywamy zalety finetuningu oraz uczenia wielozadaniowego (tab. \ref{tab:por-trening-all}). Suma czasów uczenia wyłącznie segmentacji oraz wyłącznie klasyfikacji (około 360s) znacząco przewyższa pozostałe metody. Najbardziej opłacalną czasowo metodą okazuje się finetuning. Jednakże na podstawie wyników miar jakości nie można uznać go za optymalny. Pozostają jeszcze dwie metody - nauczenie segmentacji oraz dalsze uczenie klasyfikacji (ok. 260 s . i 275 s) oraz uczenie wielozadaniowe (ok. 211s). Segmentacja, a potem klasyfikacja osiąga najlepsze wyniki na segmentacji oraz przeciętne na klasyfikacji. Z drugiej strony uczenie wielozadaniowe osiąga najlepsze rezultaty na klasyfikacji oraz drugi najlepszy wynik na segmentacji. Łącząc to z~faktem znacznie krótszego uczenia, można wysunąć wniosek, że uczenie wielozadaniowe jest optymalne pod względem czasu treningu oraz uzyskiwanych rezultatów.



% \begin{table}[ht!]
%     \centering
%     \begin{tabular}{c|c}
%         nazwa zadania                      &   czas{[}s{]} \\ \hline
%         wyłącznie segmentacja                  &   188.70 \\
%         wyłącznie klasyfikacja                &   170.47 \\
%         pośrednia klasyfikacja z segmentacji   &   70.78 \\
%         bezpośrednia klasyfikacja z segmentacji   &   88.69 \\
%         finetuning                        &   158.46 \\
%         uczenie wielozadaniowe                   &   210.97 
% \end{tabular}
% \caption{Porównanie czasu uczenia względem metody.}
% \label{tab:acc-por}
% \end{table}


\begin{table}[ht!]
    \centering
    \begin{tabular}{c|c}
        nazwa zadania                      &   czas{[}s{]} \\ \hline
        wyłącznie segmentacja +  wyłącznie klasyfikacja & \textasciitilde 360 \\
        wyłącznie segmetnacja + pośrednia klasyfikacja & \textasciitilde 260 \\
        wyłącznie segmetnacja + bezpośrednia klasyfikacja & \textasciitilde 275 \\
        uczenie wielozadaniowe                   &   \textasciitilde 211 \\
        finetuning                        &   \textasciitilde 160 
\end{tabular}
\caption{Porównanie czasu uczenia względem całości.}
\label{tab:por-trening-all}
\end{table}


Porównanie czasu wnioskowania jest kluczowe z punktu widzenia korzystania z potencjału uczenia maszynowego. Tabela \ref{tab:por-infer} przedstawia zestawienie czasu wnioskowania dla zestawu dwóch szeregowych sieci oraz jednej architektury wykonującej dwa zadania naraz. Pierwszy przypadek obejmuje wykorzystanie wyłącznie klasyfikacji, a następnie wyłącznie segmentacji. W praktyce oznacza to dwukrotne podawanie danych do modelu i przejście przrz 2 razy więcej parametrów niż w pozostałych przypadkach. Rozważając jedną architekturę osiągamy prawie dwukrotnie krótszy czas wnioskowania. Do tego przypadku zaliczamy wszystkie metody z uczeniem klasyfikacji na segmentacji, finetung oraz uczenie wielozadaniowe.


\begin{table}[ht!]
    \centering
    \begin{tabular}{c|c}
        rodzaj sieci                      &   czas{[}s{]} \\ \hline
        dwie szeregowe sieci                  &   15.7\\
        jedna architektura               &   8.6
\end{tabular}
\caption{Porównanie czasu wnioskowania.}
\label{tab:por-infer}
\end{table}

\subsection{Analiza konkretnych przykładów}
Analiza metryk, czy różnych miar jakości jest niezbędna do ewaluacji zadań uczenia maszynowego. Odpowiedni wybór tych miar gwarantuje pełen informacji wgląd, stanowiąc cenny wskazówki ewaluacyjne. Nie mniej nie wyklucza to istoty sprawdzenia rezultatów przez ludzkie oko. Mimo że trudno byłoby przeglądać i ewaluować wiele zdjęć w dużych zbiorach danych, przekrojowe sprawdzenie jest kluczowe w analizie. Dostarcza bowiem wielu cennych, nieujętych w matematycznych formułach obserwacji. W tym podrozdziale przedstawione zostaną rezultaty na wybranych zdjęciach.
\subsubsection{Segmentacja semantyczna}
Segmentacja semantyczna jest zadaniem niewątpliwie trudnym. Jednocześnie równie ciężko jest określić dobrą funkcję jakości, uwzględniającą takie właściwości jak gładkość, dokładność czy precyzja segmentacji. Można oczywiście korzystać z wielu funkcji jakości, jednak ostateczny werdykt warto przejrzeć ręcznie. W połączeniu z wiedzą dotyczącą między innymi trudności klasyfikacji danej grupy pikseli lub niejedoznacznością niektórych grup pikseli po obejrzeniu nawet kilkunastu zdjęć jesteśmy w stanie wysnuć pewne wnioski.

\noindent
\textbf{Łazienka}

Analizując rysunek \ref{fig:bathroom-pred-1} widzimy, że klasa przedmioty (ang. objects) jest bardzo szeroko rozumiana przez twórców zbioru danych. Wynika z tego fakt, że grupa ta nie posiada ściśle określonych cech, które byłyby łatwo identyfikowalne. Model w~tym przypadku połączył w sposób szeroki omawianą klasę. Ciekawą obserwacją jest zaznaczenie przez model klasy krzesło. Po głębszej analizie można przypuszczać, że zlew ma podobna teksturę oraz kształt to metalowego krzesła. Klasy ściana, podłoga oraz meble zostały dość precyzyjnie sklasyfikowane.

\begin{figure}[ht!]
    \centering
    \includegraphics[width=\textwidth]{img/preds_analysis/gt_vs_pred/bathroom-1.png}
    \caption{Porównanie jakości segmentacji dla klasy łazienka.}
    \label{fig:bathroom-pred-1}
\end{figure}

Sytuacja jest równie interesująca w przypadku rysunku \ref{fig:bathroom-pred-2}. Model dopatruje się klasy meble w okolicach drzwi oraz przy zlewie. Pierwszy przypadek jest całkiem zrozumiały. Drewniane drzwi co do faktury mogą przypominać meble, na przykład drzwi od szafki. W drugiej sytuacji można domniemywać, że meble były często związane z umywalką czy nawet zlewem kuchennym, stąd model chętnie te klasy przydziela. Interesujące jest przydzielenie przez model etykiety obraz do włącznika światła.

\begin{figure}[ht!]
    \centering
    \includegraphics[width=\textwidth]{img/preds_analysis/gt_vs_pred/bathroom-2.png}
    \caption{Porównanie jakości segmentacji dla klasy łazienka.}
    \label{fig:bathroom-pred-2}
\end{figure}

Ostatnim obraz, przedstawiający łazienkę pokazuje rysunek \ref{fig:bathroom-pred-3}. Tak jak wcześniej wspomniano ściany oraz podłoga są często dobrze klasyfikowane. Tak też jest w tym przypadku. Kosz na pranie okazał się wyzwaniem. Model doszukiwał się tu takich obiektów jak stół, krzesło czy mebel.

\begin{figure}[ht!]
    \centering
    \includegraphics[width=\textwidth]{img/preds_analysis/gt_vs_pred/bathroom-3.png}
    \caption{Porównanie jakości segmentacji dla klasy łazienka.}
    \label{fig:bathroom-pred-3}
\end{figure}

\noindent
\textbf{Salon}

Salon jest najczęściej reprezentowany przez duży pokój, w którym znajdują się kanapa, stolik z przedmiotami, krzesła/fotele oraz ściany z zawieszonymi obrazkami. Nie brakuje tutaj mebli i wielu obiektów.

Rysunek \ref{fig:living_room-pred-1} jest przykładem częstego problemu adnotacji zdjęć. Często okazuje się, że dana grupa pikseli przedstawia więcej niż jedną klasę. Obraz oczekiwany przedstawia regał z książkami jako mebel. Model stwierdził jednak wcale się nie myląc, że są to książki. Trudno się nie zgodzić z tą predykcją. Oznacza to, że zbiór jest poniekąd wewnętrznie sprzeczny w jakiejś części. Widać, że częstym kłopotem jest odróżnienie mebli od stołu. Zadowala fakt pierwszoplanowej kanapy, która bez poduszek została bardzo dobrze sklasyfikowana. Równie dobre rezultaty otrzymujemy dla klasy podłoga, sufit oraz obrazy. Dziwi natomiast fakt zaznaczenia fotela jako krzesła.

\begin{figure}[ht!]
    \centering
    \includegraphics[width=\textwidth]{img/preds_analysis/gt_vs_pred/living_room-1.png}
    \caption{Porównanie jakości segmentacji dla klasy salon.}
    \label{fig:living_room-pred-1}
\end{figure}

Na kolejnym rysunku \ref{fig:living_room-pred-2} sytuacja jest nieco gorsza. Model miał trudności ze wskazaniem stolika na środku, który klasyfikuje jako część kanapy. Sporo problemów wygenerowała klasa krzesło. Obrazy, ściany i podłoga zostały poprawnie sklasyfikowane. Przedmioty drugoplanowe, szczególnie dalsze, a więc mniejsze pozostały dla modelu jednakie.
\begin{figure}[ht!]
    \centering
    \includegraphics[width=\textwidth]{img/preds_analysis/gt_vs_pred/living_room-2.png}
    \caption{Porównanie jakości segmentacji dla klasy salon.}
    \label{fig:living_room-pred-2}
\end{figure}

Scena salonu (rys. \ref{fig:living_room-pred-3}) jest znacznie lepiej sklasyfikowana niż poprzednia. Oprócz całkiem dobrze sklasyfikowanego stołu, obiektów i kanapy jest jeden ciekawy aspekt. Mianowicie obrazy docelowe znajdujące sie w głąb obrazu zostały pominięte, czyli przedstawione jako pusta (ang. void). Mimo to klasyfikator celnie nadaje im klasy ściana oraz okna. To bardzo dobry prognostyk.

\begin{figure}[ht!]
    \centering
    \includegraphics[width=\textwidth]{img/preds_analysis/gt_vs_pred/living_room-3.png}
    \caption{Porównanie jakości segmentacji dla klasy salon.}
    \label{fig:living_room-pred-3}
\end{figure}

\noindent
\textbf{Sypialnia}

Sypialnia to miejsce bardzo złożone. Jednak do charakterystycznych punktów tej sceny należą: łóżko, krzesło, meble oraz okno. Pożądanym byłoby zatem osiągać na tych klasach satysfakcjonujące rezultaty. Na pierwszym planie rysunku \ref{fig:bedroom-pred-1} widać krzesło, stół oraz szafkę, które w przybliżeniu zostały całkiem dobrze sklasyfikowane. Brak zastrzeżeń budzą również klasy łóżko, podłoga oraz obiekty. Niewątpliwe ciekawe jest poprawne zaznaczenie okna, nawet w porze nocy. Jest to szczególnie cenna informacja, bo czarny prostokąt mógłby być zaklasyfikowany jako na przykład telewizor. Okno zostało zaznaczone zbyt szeroko, mianowicie fałszywie uznając prawdopodobnie lampę za okno. Prawdopodobnie kolor miał tu duże znaczenie.

\begin{figure}[ht!]
    \centering
    \includegraphics[width=\textwidth]{img/preds_analysis/gt_vs_pred/bedroom-1.png}
    \caption{Porównanie jakości segmentacji dla klasy sypialnia.}
    \label{fig:bedroom-pred-1}
\end{figure}

Rysunek \ref{fig:bedroom-pred-2} to typowe zdjęcie sypialni. Czarną ramę łóżka model uznał za telewizor. Gdyby wyciąć samą tę ramę, wybór rzeczywiście nie byłby oczywisty. Poza tym łózko zostało oznaczone całkiem poprawnie. Obrazy zostały poprawnie oznaczone. Na zdjęciu widać, całkiem poprawną próbę klasyfikacji krzesła.

\begin{figure}[ht!]
    \centering
    \includegraphics[width=\textwidth]{img/preds_analysis/gt_vs_pred/bedroom-2.png}
    \caption{Porównanie jakości segmentacji dla klasy sypialnia.}
    \label{fig:bedroom-pred-2}
\end{figure}

Ostatni rysunek (rys. \ref{fig:bedroom-pred-3}) był większym wyzwaniem dla modelu. Widać to szczególnie w przypadku pierwszoplanowego biurka. Model nie był w stanie podjąć decyzji co do ostatecznej klasy. Standardowo podłoga oraz sufit zostały sklasyfikowane prawidłowo. Nie inaczej było w przypadku klasy łózko.
\begin{figure}[ht!]
    \centering
    \includegraphics[width=\textwidth]{img/preds_analysis/gt_vs_pred/bedroom-3.png}
    \caption{Porównanie jakości segmentacji dla klasy sypialnia.}
    \label{fig:bedroom-pred-3}
\end{figure}

\noindent
\textbf{Jadalnia}


Obrazy związane z jadalnią to głównie sceny związane ze stołami oraz krzesłami.

Taka sytuacja ma też miejsce na rysunku \ref{fig:dining_room-pred-1}. Właściwie trudno tutaj znaleźć coś szczególnie interesującego. Cały obraz został całkiem dobrze pogrupowany. Wątpliwości budzi jedynie przypisanie do żyrandola klasy obrazy. Prawdopodobnie obrazy znajdujące się obok miały na to wpływ.


\begin{figure}[ht!]
    \centering
    \includegraphics[width=\textwidth]{img/preds_analysis/gt_vs_pred/dining_room-1.png}
    \caption{Porównanie jakości segmentacji dla klasy jadalnia.}
    \label{fig:dining_room-pred-1}
\end{figure}

Przypadek rysunku \ref{fig:dining_room-pred-2} wydaje się ciekawszym. Szczególnie warte uwagi są tutaj okna, na których znajdują się odbicia lustrzane. Refleksy są w wizji komputerowej zagadnieniem od dawna poruszanym i znanym. Można jednoznacznie stwierdzić, że trudno sobie poradzić w takich sytuacjach. Model prawdopodobnie mając trudności z tym obszarem, przypisał go do klasy obiekt. Oprócz tego widzimy problemy z krzesłami w prawym dolnym rogu. Jasna, połyskująca skóra rzeczywiście przypomina nieco płytki podłogowe.

\begin{figure}[ht!]
    \centering
    \includegraphics[width=\textwidth]{img/preds_analysis/gt_vs_pred/dining_room-2.png}
    \caption{Porównanie jakości segmentacji dla klasy jadalnia.}
    \label{fig:dining_room-pred-2}
\end{figure}

Ostanim analizowany obraz w jadalni jest rysunek \ref{fig:dining_room-pred-3}. Na pewno klasyfikacja klas takich jak stół, krzesła czy okno jest tutaj poprawna. Co więcej nie można tego do tego grona niezaliczyć klasy podłoga oraz sufit. Jedyny problem z grupowaniem na tym zdjęciu dotyczy samego roku zdjęcia, gdzie nie przyporządkowano klasy mebel. Pozostałe instacje tej klasy są poprawnie sklasyfikowane.

\begin{figure}[ht!]
    \centering
    \includegraphics[width=\textwidth]{img/preds_analysis/gt_vs_pred/dining_room-3.png}
    \caption{Porównanie jakości segmentacji dla klasy jadalnia.}
    \label{fig:dining_room-pred-3}
\end{figure}

\noindent
\textbf{Kuchnia}

Obrazy przedstawiające kuchnie to głównie zabudowa kuchni oraz sprzęt kuchenny. Czasen występuje tutaj na przykład stół z krzesłami.

Obraz \ref{fig:kitchen-pred-1} nie wydaje się trudnym do klasyfikacji, jednak pojawiło się tutaj kilka kwestii wartych omówienia. Oprócz problemów z klasyfikacją stołu z prawej strony, który bardziej wygląda jak szafki z blatem w kuchni, obserwujemy błędne przypisanie tapety naściennej jako obrazy. Poza tym drewniane drzwi model klasyfikuje jako bardziej mebel niż ścianę, co ze względu na teksturę nie jest aż tak złym wyborem. Reszta zdjęcia została pogrupowana poprawnie.

\begin{figure}[ht!]
    \centering
    \includegraphics[width=\textwidth]{img/preds_analysis/gt_vs_pred/kitchen-1.png}
    \caption{Porównanie jakości segmentacji dla klasy kuchnia.}
    \label{fig:kitchen-pred-1}
\end{figure}

Na rysunku \ref{fig:kitchen-pred-2} widzimy typową wąską kuchnię. Rezultaty są w miarę zadowalające poza przypisaniem lodówki do klasy obraz. Prawdopodbnie miały na to wpływ zdjęcia zawieszone na lodówce. Ściany, szafki i sufit zostały zaklasyfikowany prawidłowo.

\begin{figure}[ht!]
    \centering
    \includegraphics[width=\textwidth]{img/preds_analysis/gt_vs_pred/kitchen-2.png}
    \caption{Porównanie jakości segmentacji dla klasy kuchnia.}
    \label{fig:kitchen-pred-2}
\end{figure}

Trzecim rysunkiem jest rys. \ref{fig:kitchen-pred-3}. Największe wyzwanie stanowią tutaj obiekty zlokalizowane w różnych miejscach. Cieszy fakt, że mimo iż autorzy błędnie ocenili krzesło jako obiekt, model i tak zaznaczył je poprawnie. Widzimy tutaj również próbę klasyfikacji stołu. Powraca wtedy dyskusja na temat czy stół jest meblem tak jak został zaklasyfikowany przez model.

\begin{figure}[ht!]
    \centering
    \includegraphics[width=\textwidth]{img/preds_analysis/gt_vs_pred/kitchen-3.png}
    \caption{Porównanie jakości segmentacji dla klasy kuchnia.}
    \label{fig:kitchen-pred-3}
\end{figure}

\noindent
\textbf{Biuro}

Sceny związane z biurem najczęściej przedstawiają biurka z krzesłami, zarówno w~faktycznych biurach, o których często świadczy wykładzina, jak i w domowych pokojach typu biuro.

Na rysunku \ref{fig:office-pred-1} widać scenę przedstawiające pokój z drukarkami. Model dość dobrze radzi sobie ze ścianami oraz z podłogą, której akurat w tym przypadku nie ma zbyt wiele. Ciekawa jest wizja autorów zbioru danych określających mapę jako obiekt zamiast obraz. Może trudno bez wahania przypisać wiszącej mapie miano obrazu, ale na pewno szybciej możną ją określić jako plakat co można tłumazyć na angielski jako picture.

\begin{figure}[ht!]
    \centering
    \includegraphics[width=\textwidth]{img/preds_analysis/gt_vs_pred/office-1.png}
    \caption{Porównanie jakości segmentacji dla klasy biuro.}
    \label{fig:office-pred-1}
\end{figure}

Rysunek \ref{fig:office-pred-2} przedstawia salę konferencyjną. Lewa strona obrazu została zaklasyfikowana całkiem poprawnie. Wyzwaniem dla modelu okazał się prawy dolny róg, gdzie należało przypisać klasy mebel, obiekt, ściana, co model uprościł do po prostu mebla. To zdecydowanie zła klasyfikacja.

\begin{figure}[ht!]
    \centering
    \includegraphics[width=\textwidth]{img/preds_analysis/gt_vs_pred/office-2.png}
    \caption{Porównanie jakości segmentacji dla klasy biuro.}
    \label{fig:office-pred-2}
\end{figure}

Ostanim rysunkiem {\ref{fig:office-pred-3}} jest pomieszczenie przedstawiające najprawdopodobniej biuro domowe. Klasyfikacja okna, podłogi oraz ściany była prawie bezbłędna. Gorzej model porawdził sobie z stołem, który po części sklasyfikował jako telewizor ze względu na bardzo ciemny oraz prostokąty charakter.

\begin{figure}[ht!]
    \centering
    \includegraphics[width=\textwidth]{img/preds_analysis/gt_vs_pred/office-3.png}
    \caption{Porównanie jakości segmentacji dla klasy biuro.}
    \label{fig:office-pred-3}
\end{figure}

\noindent
\textbf{Inne pomieszczenia}

Sceny związane z klasą inne pomieszczenia budzą najwięcej wątplości. Nie wiadomo bowiem, co dokładnie może się tam znaleźć.

Na rysunku \ref{fig:other_indoor-pred-1} znajduje się wspólna przesztrzeń biurowa. Widzimy, że znajduje się tutaj wiele obszarów typu void, zatem model dokładnie nie wie co powinno się tam znaleźć. Dziwi to szczególnie w przypadku pierwszego krzesła po prawej stronie. Nie mniej jednak model dość dobrze zgaduję tę klasę. Jest zrozumiałym, że pokój otoczy ścianami. W gruncie rzeczy szklana szyba rzeczywiście jest ścianą w tym przypadku. Model niezbyt dobrze pogrupował klasę obiekty. Jest tutaj wiele do poprawy.

\begin{figure}[ht!]
    \centering
    \includegraphics[width=\textwidth]{img/preds_analysis/gt_vs_pred/other_indoor-1.png}
    \caption{Porównanie jakości segmentacji dla klasy inne pomieszczenia.}
    \label{fig:other_indoor-pred-1}
\end{figure}

Rysunek \ref{fig:other_indoor-pred-2} przedstawia pomieszczenie biurowe. Wszystkie obrazki zostały zaklasyfikowane poprawnie. Okna zostały przypisane jako obrazy. Model dobrze pogrupował człowieka. Wyzwanie stanowiła klasa obiekty.

\begin{figure}[ht!]
    \centering
    \includegraphics[width=\textwidth]{img/preds_analysis/gt_vs_pred/other_indoor-2.png}
    \caption{Porównanie jakości segmentacji dla klasy inne pomieszczenia.}
    \label{fig:other_indoor-pred-2}
\end{figure}


Ostatnim analizowanym obrazem w jadalni jest rysunek \ref{fig:dining_room-pred-3}. Na pewno klasyfikacja stółu, krzesła czy okna jest tutaj poprawna. Co więcej, nie można tego do tego grona nie zaliczyć klasy podłoga oraz sufit. Jedyny problem z grupowaniem na tym zdjęciu dotyczy samego rogu zdjęcia, gdzie nie przyporządkowano klasy mebel. Pozostałe instancje tej klasy są poprawnie sklasyfikowane.

\begin{figure}[ht!]
    \centering
    \includegraphics[width=\textwidth]{img/preds_analysis/gt_vs_pred/other_indoor-3.png}
    \caption{Porównanie jakości segmentacji dla klasy inne pomieszczenia.}
    \label{fig:other_indoor-pred-3}
\end{figure}

\subsubsection{Klasyfikacja sceny}
Podobnie jak w przypadku segmentacji semantycznej czasem trudno jest jednoznacznie określić jakość modelu, bazując wyłącznie na miarach jakości. W niniejszym rozdziale zostaną przytoczone wszystkie błędne klasyfikacje z podziałem na konkretne klasy. Pozwoli to wysnuć pewne obserwacje na temat podobieństw tych klas oraz pomoże wysunąć wnioski co do tych błędów. Co więcej, przedstawione zostaną statystyki błędnej klasyfikacji, aby lepiej zobrazować te błędy.

Na rysunku \ref{fig:bathroom-false-pred} przedstawiono 10 błędnych przypisań dla klasy łazienka. Dziewięć z dziesięciu błędów dotyczyło klasy kuchnia. Można doszukiwać się, że kuchnia, jak i~łazienka ma poniekąd podobny schemat. Na pewno występuję te same klasy jak zlew czy meble. Tylko raz klasyfikator uznał, że sypialnia jest łazienką.
\begin{figure}[ht!]
    \centering
    \includegraphics[width=\textwidth]{img/preds_analysis/classification/bathroom-2.png}
    \caption{Porównanie jakości klasyfikacji dla klasy łazienka.}
    \label{fig:bathroom-false-pred}
\end{figure}

Wiele pomyłek algorytm popełnił na klasie sypialnia (rys.\ref{fig:bedroom-false-pred}). Na pewno wynika to z faktu, iż była to klasa dominująca. Co drugi błąd następował na klasie sypialnia. Szczególnie często gdy łózkiem była kanapa lub na zdjęciu występował fotel. Ponad 32\% błędów w sumie stanowiły klasy biuro oraz jadalnia. Można przypuszczać, że tym razem kluczowym elementem świadczącym o predykcji był stół.
\begin{figure}[ht!]
    \centering
    \includegraphics[width=\textwidth]{img/preds_analysis/classification/bedroom-2.png}
    \caption{Porównanie jakości klasyfikacji dla klasy sypialnia.}
    \label{fig:bedroom-false-pred}
\end{figure}

Jadalnia była pomylona w sumie 11 razy (rys. \ref{fig:dining_room-false-pred}). Zgodnie z przedstawionym rysunkiem, na większości zdjęć występuje stół i krzesła.

\begin{figure}[ht!]
    \centering
    \includegraphics[width=\textwidth]{img/preds_analysis/classification/dining_room-2.png}
    \caption{Porównanie jakości klasyfikacji dla klasy jadalnia.}
    \label{fig:dining_room-false-pred}
\end{figure}

Najmniej pomyłek jest dla klasy kuchnia (rys. \ref{fig:kitchen-false-pred}). Dwa z czterech błędów dotyczy jadalni, a więc pomieszczenia często należącego kuchni. Stąd mogą wynikać rozbieżności.
\begin{figure}[ht!]
    \centering
    \includegraphics[width=\textwidth]{img/preds_analysis/classification/kitchen-2.png}
    \caption{Porównanie jakości klasyfikacji dla klasy kuchnia.}
    \label{fig:kitchen-false-pred}
\end{figure}

Zdecydowanie najwięcej razy model pomylił się dla klasy salon (rys. \ref{fig:living_room-false-pred}). 35\% błędów należy do klasy sypialnia. Trochę mniej, bo 25\% pomyłek to klasa biuro. Reszta klas to około 15\% błędów zarówno dla innych pomieszczemieszczeń jak i jadalni. Nieznaczną ilość razy (8\%) model przyporządkował klasie kuchnia. Widać, że zdjęcia sypialni najczęści są bardzo podbne do salonu. Zdecydowana większość obrazów zawiera kanapę, stąd może to być zwodnicze.

\begin{figure}[ht!]
    \centering
    \includegraphics[width=\textwidth]{img/preds_analysis/classification/living_room-2.png}
    \caption{Porównanie jakości klasyfikacji dla klasy salon.}
    \label{fig:living_room-false-pred}
\end{figure}

Klasy inne pomieszczenia oraz kuchnie zostały błędnie zaklasyfikowane jako biuro w~większości przypadków (rys. \ref{fig:office-false-pred}). Trudno określić, skąd akurat wynikają takie rezultaty.

\begin{figure}[ht!]
    \centering
    \includegraphics[width=\textwidth]{img/preds_analysis/classification/office-2.png}
    \caption{Porównanie jakości klasyfikacji dla klasy biuro.}
    \label{fig:office-false-pred}
\end{figure}

Model najczęściej błędnie przypisywał klasę inne pomieszczenia dla biura w mniej niż połowie przypadków (rys. \ref{fig:oter_indoor-false-pred}). Pozostałe przypadki należą do klas sypialnia oraz jadalnia. Klasa inne pomieszczenia jest szczególnie narażona na pomyłki, gdyż to połączenie najróżniejszych klas scen.
\begin{figure}[ht!]
    \centering
    \includegraphics[width=\textwidth]{img/preds_analysis/classification/other_indoor-2.png}
    \caption{Porównanie jakości klasyfikacji dla klasy inne pomieszczenia.}
    \label{fig:oter_indoor-false-pred}
\end{figure}


Analiza błędnie sklasyfikowanych scen dostarczyła wielu ważnych informacji. Najczęściej przyczyną błędów było znaczne podobieństwo występujących klas przedmiotów między różnymi klasami scen.


    
    % \begin{figure}[ht!]
    %     \centering
    %     \includegraphics[width=0.75\textwidth]{scene_comp.png}
    %     \caption{Porównanie dokładności dla każdej z klas w zadaniu klasyfikacji pomieszczeń}
    %     \label{fig:scene_comp}
    % \end{figure}
    
    % \begin{figure}[ht!]
    %     \centering
    %     \includegraphics[width=0.75\textwidth]{seg_comp.png}
    %     \caption{Porównanie dokładności dla każdej z klas w zadaniu segmentacji semantycznej}
    %     \label{fig:seg_comp}
    % \end{figure}
    % Ucznie wielozadaniowe w rozważanym przypadku nieznacznie poprawia wyniki sieci (tab. \ref{tab:acc-por}). Dla zadania segmentacji semantycznej otrzymujemy spadek jakości o 0.39 punkta procentowego. Zadanie klasyfikacji poprawia się o 1.95 p.p. w porównaniu z uczeniem jednozadaniowym. Ostatecznie otrzymujemy zysk na poziomie 0.78 punkta procentego na średniej z zadań. Poprawa jest niewielka, jednak jest to dużu sukces biorąc pod uwagę, że mamy do dyspozycji 2 razy mniej parametrów niz w przypadku dwóch osobnych sieci. Przekłada się to bezpośrednio na czas inferencji, który w przypadku robotyki i systemów czasu rzeczywistego jest kluczowy.
    
    % Wartym zobaczenia jest fakt, iż uczenie wielozadaniowe poprawia wyniki dla klas które osiągają najsłabsze rezultaty w uczeniu jednozadaniowym. Poprawie ulega klasa office (rys. \ref{fig:scene_comp}) dla klasyfikacji oraz klasy books oraz table (rys. \ref{fig:seg_comp}) dla segmentacji semantycznej. Powodem jest prawdopodbnie mniejsze obciążenie (bias) modelu spowodowane faktem wzajemnej regularyzacji obu zadań w procesie uczenia. Innymi słowy, model ma mniejszą tendencję do przeuczenia.
\newpage % Rozdziały zaczynamy od nowej strony
\newpage % Rozdziały zaczynamy od nowej strony.
\section{Podsumowanie}

-mozna bylo lepiej to zrobic:
    - augmentacja
    - więcej epok
    - regularyzacja smoothingiem
    - regularayzacja weight decay
-zeby w pelni moc ocenic potencjal uczenia wielozadaniwoe nalezaloby spawdzic modele na wielu zbiorach, w razie potrzeby zachowujac przestrzen reprezentacji a dosziflowujac ostatnie warstwy decyzyjbedroomeabedroom



niepoprawen klasy szum podczas uczenia
klasa obiekt generuja problemy klasy szerokie generuja problemy
Analiza przewidywań dostarczyła wielu cennych szczegółów, które byłoby trundo zauważyć patrząc jedynie na liczby.
drewno to meble
obrazy zawsze były na powierzchniach takich jak sciany czy szafki
REZULATAT
słabe wyniki bo
- sprzecznośc klas (regał mebel czy ksiązki, czy mebel i stół)
- klasa objects

%--------------------------------------------
% Literatura
%--------------------------------------------
\cleardoublepage % Zaczynamy od nieparzystej strony
\printbibliography

%--------------------------------------------
% Spisy (opcjonalne)
%--------------------------------------------
\newpage
\pagestyle{plain}

% Wykaz symboli i skrótów.
% Pamiętaj, żeby posortować symbole alfabetycznie
% we własnym zakresie. Ponieważ mało kto używa takiego wykazu,
% uznałem, że robienie automatycznie sortowanej listy
% na poziomie LaTeXa to za duży overkill.
% Makro \acronymlist generuje właściwy tytuł sekcji,
% w zależności od języka.
% Makro \acronym dodaje skrót/symbol do listy,
% zapewniając podstawowe formatowanie.
% //AB
% \vspace{0.8cm}
% \acronymlist
% \acronym{EiTI}{Wydział Elektroniki i Technik Informacyjnych}
% \acronym{PW}{Politechnika Warszawska}
% \acronym{WEIRD}{ang. \emph{Western, Educated, Industrialized, Rich and Democratic}}

\listoffigurestoc     % Spis rysunków.
\vspace{1cm}          % vertical space
\listoftablestoc      % Spis tabel.
% \vspace{1cm}          % vertical space
% \listofappendicestoc  % Spis załączników

% Załączniki
% \newpage
% \appendix{Nazwa załącznika 1}
% \lipsum[1-8]

% \newpage
% \appendix{Nazwa załącznika 2}
% \lipsum[1-4]

% Używając powyższych spisów jako szablonu,
% możesz tu dodać swój własny wykaz bądź listę,
% np. spis algorytmów.

\end{document} % Dobranoc.
